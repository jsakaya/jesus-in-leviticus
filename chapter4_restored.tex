\chapter{Christ in the Day of Atonement}

\begin{scripture}
``Not with the blood of goats and calves, but with His own blood He entered the Most Holy Place once for all, having obtained eternal redemption.''
\scriptureref{Hebrews 9:12}
\end{scripture}

\textit{Theme: The once-a-year atonement of Israel is fulfilled in the once-for-all atonement of Christ.}

\section{Introduction: The Pinnacle of Leviticus}

Among Exodus's thirty-nine chapters, Leviticus's twenty-seven, and the journey through Numbers and Deuteronomy, one chapter rises like the \textbf{pinnacle of a holy mountain} --- \textbf{Leviticus 16}, the Day of Atonement.

If the book of Leviticus is a sanctuary, then chapter 16 is the \textbf{Holy of Holies}. If Leviticus is a mountain, then chapter 16 is the \textbf{summit}. If Leviticus is a lens, then chapter 16 is its \textbf{clearest focus}.

Every law, every offering, every priestly function, every clean-unclean distinction ascends toward this central chapter --- the one day when \textbf{one man} entered \textbf{one place} with \textbf{one offering}, for \textbf{one nation}, pointing to the final work of \textbf{One Perfect Christ} for \textbf{all nations}.

\section{Meaning of ``Atonement''}

The Hebrew word \heb{כִּפֵּר} (\textit{kāphar}) means ``to cover,'' ``to make reconciliation,'' or ``to make satisfaction'' --- the three dimensions of true atonement. It comes from a root meaning ``to smear or cover over.'' In the Old Testament, sin was \textit{covered} temporarily; in the New Testament, through Christ, sin is \textit{removed} eternally.

\subsection{Distinction}

\begin{itemize}
\item \textbf{Old Covenant:} Sin covered until Christ (Rom.\ 3:25 --- ``God passed over the sins previously committed'').
\item \textbf{New Covenant:} Sin taken away forever (John 1:29 --- ``Behold, the Lamb of God who takes away the sin of the world'').
\end{itemize}

Thus, Leviticus 16 shows the \textbf{shadow}, while Hebrews 9--10 shows the \textbf{substance}. The blood of bulls and goats could cover sin; only the blood of the Son could cancel it.

\section{The Central Day in Israel's Calendar}

The Day of Atonement (\heb{יוֹם הַכִּפֻּרִים} \textit{Yom Kippur}) was the holiest and most solemn day in Israel's entire year. It occurred on the tenth day of the seventh month (Tishri), following the Feast of Trumpets and preceding the Feast of Tabernacles (Lev.\ 23:26--32). Unlike other days of feasting and joy, this was a day of affliction, repentance, and cleansing.

\begin{itemize}
\item Every Israelite was commanded to ``afflict his soul'' (Lev.\ 23:27).
\item All work ceased.
\item The nation stood still as their high priest entered the Holy of Holies once a year (Heb.\ 9:7).
\end{itemize}

It was the only time the veil was passed, the mercy seat approached, and the blood carried within the veil. Everything about the day declared: ``Access to God is not open except through blood, and through a mediator.''

\subsection{Purpose of the Day}

\begin{itemize}
\item To cleanse the sanctuary from the defilement caused by the sins of the people (Lev.\ 16:16).
\item To reconcile the people to God (v.30).
\item To renew fellowship between a holy God and a sinful nation.
\end{itemize}

In shadow, it anticipated the one eternal atonement accomplished at the Cross.

\section{The Three Main Participants}

\subsection{The High Priest (Aaron)}

\begin{itemize}
\item Represented the entire nation before God.
\item On that day, he laid aside his golden garments and wore simple white linen (Lev.\ 16:4), symbolizing humility and purity.
\item Type of Christ, who ``made Himself of no reputation'' and took ``the form of a servant'' (Phil.\ 2:7).
\end{itemize}

Aaron entered the Holy of Holies with fear and trembling; Christ entered heaven itself in triumph and glory.

\subsection{The Two Goats}

\begin{itemize}
\item They formed one offering for sin (Lev.\ 16:5).
\item One was slain, the other sent away --- two aspects of one atonement:
  \begin{itemize}
  \item The slain goat: Sin judged before God.
  \item The scapegoat: Sin removed from man's sight.
  \end{itemize}
\end{itemize}

\subsection{The People}

\begin{itemize}
\item They waited outside the sanctuary, depending wholly on the priest's success.
\item Their peace and purity rested on whether the atonement was accepted.
\item In Christ, believers are no longer waiting outside but have ``boldness to enter the Holiest by the blood of Jesus'' (Heb.\ 10:19).
\end{itemize}

\section{Key Actions on the Day}

\begin{table}[h]
\centering
\begin{tabular}{clc}
\toprule
\textbf{Stage} & \textbf{Action} & \textbf{Reference} \\
\midrule
1 & Preparation of the high priest (washing, linen garments) & 16:3--4 \\
2 & Sin offering for himself (bull) & 16:6 \\
3 & Sin offering for the people (two goats) & 16:7--10 \\
4 & Entrance into the Most Holy Place & 16:11--14 \\
5 & Sprinkling of blood on the mercy seat & 16:15--16 \\
6 & Confession over the scapegoat & 16:20--22 \\
7 & Sending away the scapegoat into the wilderness & 16:21--22 \\
8 & Burning of the carcasses outside the camp & 16:27 \\
9 & Completion and rest & 16:29--31 \\
\bottomrule
\end{tabular}
\end{table}

\section{The High Priest as Type of Christ}

\subsection{His Preparation (vv. 3--4)}

He washed with water and put on plain linen garments, laying aside his usual robe of glory and beauty. \textbf{Typology:} Christ, though divine, laid aside His glory to humble Himself and take the form of a servant (Phil.\ 2:7--8; John 17:19).

\subsection{His Personal Sin Offering (v. 6)}

Aaron offered a bull for himself --- showing that Levitical priests were sinners. Before representing the people, Aaron needed cleansing; Christ, sinless, represents us perfectly.

\textbf{Contrast:} Christ needed no such offering --- ``He knew no sin'' (2 Cor.\ 5:21; Heb.\ 7:26--27).

\subsection{The Slain Goat --- Blood Sprinkled (vv. 15--19)}

The goat was killed, and its blood was carried within the veil. The blood was sprinkled upon and before the mercy seat --- seven times. \textbf{Type:} Christ's blood presented before the throne of God in heaven.

\begin{scripture}
``Not with the blood of goats and calves, but with His own blood He entered the Most Holy Place once for all.''
\scriptureref{Hebrews 9:12}
\end{scripture}

\subsection{His Entrance within the Veil (vv. 11--14)}

He entered the Most Holy Place carrying blood, surrounded by a cloud of incense lest he die. \textbf{Typology:} Christ entered ``into heaven itself, now to appear in the presence of God for us'' (Heb.\ 9:24).

\subsection{The Two Goats: Dual Aspect of Atonement}

Two goats were taken for one sin offering (16:5) --- not two sacrifices but two halves of one act.

\begin{table}[h]
\centering
\small
\begin{tabularx}{\textwidth}{l>{\raggedright\arraybackslash}X>{\raggedright\arraybackslash}X>{\raggedright\arraybackslash}X}
\toprule
\textbf{Goat} & \textbf{Action} & \textbf{Meaning} & \textbf{Fulfillment in Christ} \\
\midrule
The slain goat & Blood sprinkled on the mercy seat & Propitiation --- satisfaction of God's justice & ``God set forth Christ as a propitiation'' (Rom.\ 3:25) \\
The scapegoat (\heb{עֲזָאזֵל} \textit{ʿAzāzēl}) & Bearing confessed sins into the wilderness & Expiation --- removal of sin from the people & ``The Lamb of God who takes away the sin of the world'' (John 1:29) \\
\bottomrule
\end{tabularx}
\end{table}

Thus one goat represents Godward satisfaction, the other manward removal --- both fulfilled in the Cross.

\subsection{The Blood Sprinkled Seven Times}

\textit{``He shall sprinkle it upon the mercy seat and before the mercy seat seven times.''} (Lev.\ 16:14)

Seven signifies completion. Every drop testified that atonement was fully accomplished.

\begin{scripture}
``He entered once for all into the holy places\ldots by means of His own blood, thus securing eternal redemption.''
\scriptureref{Hebrews 9:12}
\end{scripture}

\subsection{The Scapegoat Sent Away}

Aaron laid both hands on the live goat, confessing all the sins of Israel. The goat was sent away into a desolate land, symbolically carrying their guilt far from them.

\textbf{Fulfilment:}
\begin{scripture}
``As far as the east is from the west, so far has He removed our transgressions from us.''
\scriptureref{Psalm 103:12}
\end{scripture}

\textit{``He bore our sins in His own body on the tree.''} (1 Pet.\ 2:24)

\textbf{Spiritual meaning:} The first goat satisfies divine justice. The second removes human guilt. Together they portray the twofold result of Calvary: forgiveness before God and cleansing of conscience.

\subsection{The Burning Outside the Camp}

\textit{``The bull and the goat whose blood was brought in to make atonement\ldots shall be burned outside the camp.''} (16:27)

This points directly to Hebrews 13:11--12: \textit{``Jesus also suffered outside the gate in order to sanctify the people through His own blood.''}

The burning outside symbolizes rejection, shame, and complete consumption of sin --- all borne by Christ on the Cross.

\subsection{The Sabbath of Rest}

The day concluded with absolute rest (16:29--31). Israel was commanded to do no work, for atonement was entirely the priest's task.

Typologically, salvation is by grace alone: ``He sat down'' (Heb.\ 10:12). ``We who have believed enter into rest'' (Heb.\ 4:3).

\section{The Annual vs. the Eternal}

\begin{table}[h]
\centering
\small
\begin{tabularx}{\textwidth}{l>{\raggedright\arraybackslash}X>{\raggedright\arraybackslash}X}
\toprule
\textbf{Contrast} & \textbf{Day of Atonement} & \textbf{Christ's Atonement} \\
\midrule
Frequency & Every year & Once for all (Heb.\ 9:26) \\
Priest & Many, mortal & One, eternal \\
Blood & Of animals & Of Christ \\
Sanctuary & Earthly, symbolic & Heavenly, real \\
Result & Temporary covering & Eternal redemption \\
Access & Restricted & Open to all believers \\
Memory of sin & Continual & Sin remembered no more (Heb.\ 10:17) \\
\bottomrule
\end{tabularx}
\end{table}

Thus, Leviticus 16 prefigures Hebrews 9--10: the same structure, but fulfilled by a sinless, immortal High Priest.

\section{The Mercy Seat and the Cross}

The \heb{כַּפֹּרֶת} (\textit{kappōret}, mercy seat) was the golden lid covering the ark of the covenant, overshadowed by cherubim. Beneath lay the tablets of the law --- God's righteous standard; above hovered His glory. Only when blood was sprinkled did judgment and mercy meet.

\begin{scripture}
``Righteousness and peace have kissed each other.''
\scriptureref{Psalm 85:10}
\end{scripture}

\begin{scripture}
``Whom God put forward as a propitiation (\textit{hilastērion}) by His blood.''
\scriptureref{Romans 3:25}
\end{scripture}

At the Cross, the blood of Christ satisfied the law beneath and revealed the glory above. The place of judgment became the throne of grace.

\section{Doctrinal Synthesis}

\begin{enumerate}
\item \textbf{Substitution} --- The innocent died for the guilty (the slain goat).
\item \textbf{Identification} --- The sins were confessed and transferred (the scapegoat).
\item \textbf{Propitiation} --- God's holiness satisfied by blood on the mercy seat.
\item \textbf{Expiation} --- Sin removed and remembered no more.
\item \textbf{Access} --- The veil opened; believers enter boldly (Heb.\ 10:19--22).
\item \textbf{Intercession} --- Christ now appears for us continually.
\item \textbf{Rest} --- The finished work invites faith, not labor.
\end{enumerate}

\begin{scripture}
``By one offering He has perfected forever those who are being sanctified.''
\scriptureref{Hebrews 10:14}
\end{scripture}

\section{Chart: The Levitical Day and the Cross}

\begin{table}[h]
\centering
\small
\begin{tabularx}{\textwidth}{l>{\raggedright\arraybackslash}X>{\raggedright\arraybackslash}X}
\toprule
\textbf{Levitical Element} & \textbf{Meaning} & \textbf{Fulfillment in Christ} \\
\midrule
High Priest & Mediator for Israel & Christ the Great High Priest \\
Linen garments & Humility & Incarnation \\
Bull for Aaron & Personal atonement & Sinless priest, no need \\
Goat for LORD & Propitiation & Christ's sacrificial death \\
Scapegoat & Expiation & Bearing away sin \\
Mercy seat & Throne of grace & Cross of Christ \\
Blood sprinkled & Satisfaction & ``It is finished.'' \\
Outside burning & Curse borne & ``Outside the gate'' (Heb.\ 13:12) \\
Rest of the day & Finished work & Believers' rest in grace \\
\bottomrule
\end{tabularx}
\end{table}

\section{Devotional Reflection}

When Aaron emerged from the Holy of Holies, the people waited in silent expectation; when he appeared, they rejoiced --- the atonement was accepted. Likewise, the disciples saw the risen Christ --- proof that God's justice was satisfied.

\begin{scripture}
``When He had made purification for sins, He sat down at the right hand of the Majesty on high.''
\scriptureref{Hebrews 1:3}
\end{scripture}

The Day of Atonement therefore speaks of Calvary (atonement accomplished), the Resurrection (atonement accepted), and the Session (atonement applied).

\section{Summary Table --- The Gospel in the Day of Atonement}

\begin{table}[h]
\centering
\small
\begin{tabularx}{\textwidth}{l>{\raggedright\arraybackslash}X>{\raggedright\arraybackslash}X}
\toprule
\textbf{Doctrinal Term} & \textbf{Levitical Type} & \textbf{Fulfillment in Christ} \\
\midrule
Substitution & Slain goat & ``The righteous for the unrighteous'' \\
Propitiation & Blood on mercy seat & ``Whom God displayed as propitiation'' \\
Expiation & Scapegoat sent away & ``Takes away the sin of the world'' \\
Mediation & High priest entering & ``One Mediator between God and men'' \\
Intercession & Incense before the veil & ``Ever lives to make intercession'' \\
Sanctification & Priestly cleansing & ``Sanctified through His own blood'' \\
Reconciliation & Atonement for all & ``Peace through the blood of His cross'' \\
\bottomrule
\end{tabularx}
\end{table}

The shadow (Leviticus 16) reveals the holiness of God and the distance between God and man. The substance (Hebrews 9) reveals the love of God and the nearness made possible through Christ. The Day of Atonement was a day of fear; Calvary made it a day of access. The High Priest in Israel had to exit; our High Priest never leaves the presence of God.
