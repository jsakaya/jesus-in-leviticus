\documentclass[11pt,a4paper,openany]{book}

% ========== PACKAGES (Load before polyglossia/bidi) ==========
\usepackage[utf8]{inputenc}
\usepackage[T1]{fontenc}
\usepackage{fontspec}
\usepackage{xcolor}
\usepackage{graphicx}

\usepackage{geometry}
\geometry{
    a4paper,
    margin=1in,
    top=1.2in,
    bottom=1.2in
}

\usepackage{titlesec}
\usepackage{titletoc}
\usepackage{fancyhdr}
\usepackage{booktabs}
\usepackage{longtable}
\usepackage{array}
\usepackage{tabularx}
\usepackage{multirow}
\usepackage{enumitem}
\usepackage{quoting}
\usepackage{lettrine}
\usepackage{microtype}
\usepackage{setspace}
\usepackage{hyperref}

% ========== COLORS (Define before polyglossia) ==========
\definecolor{chaptercolor}{RGB}{139, 69, 19}
\definecolor{sectioncolor}{RGB}{102, 51, 0}
\definecolor{quotecolor}{RGB}{70, 70, 70}
\definecolor{tableheader}{RGB}{210, 180, 140}

% ========== HYPERREF SETUP ==========
\hypersetup{
    colorlinks=true,
    linkcolor=sectioncolor,
    urlcolor=sectioncolor,
    pdftitle={Jesus in the Book of Leviticus},
    pdfauthor={Course Handbook},
}

% ========== POLYGLOSSIA (Must be loaded last due to bidi) ==========
\usepackage{polyglossia}
\setmainlanguage{english}
\setotherlanguage{hebrew}
\newfontfamily\hebrewfont{Times New Roman}[Script=Hebrew]

% ========== CHAPTER AND SECTION FORMATTING ==========
\titleformat{\chapter}[display]
    {\normalfont\huge\bfseries\color{chaptercolor}}
    {\chaptertitlename\ \thechapter}{20pt}{\Huge}
\titlespacing*{\chapter}{0pt}{-20pt}{40pt}

\titleformat{\section}
    {\normalfont\Large\bfseries\color{sectioncolor}}
    {\thesection}{1em}{}

\titleformat{\subsection}
    {\normalfont\large\bfseries\color{sectioncolor}}
    {\thesubsection}{1em}{}

% ========== HEADER AND FOOTER ==========
\pagestyle{fancy}
\fancyhf{}
\fancyhead[LE,RO]{\thepage}
\fancyhead[RE]{\textit{Jesus in the Book of Leviticus}}
\fancyhead[LO]{\textit{\leftmark}}
\renewcommand{\headrulewidth}{0.4pt}
\renewcommand{\footrulewidth}{0pt}

% ========== QUOTE ENVIRONMENT ==========
\newenvironment{scripture}{%
    \begin{quoting}[leftmargin=1.5em, rightmargin=1.5em, vskip=0.5em]
    \itshape\color{quotecolor}
}{%
    \end{quoting}
}

% ========== HEBREW COMMAND ==========
\newcommand{\heb}[1]{\texthebrew{#1}}

% ========== DOCUMENT ==========
\begin{document}

% ========== TITLE PAGE ==========
\begin{titlepage}
    \centering
    \vspace*{2cm}

    {\Huge\bfseries\color{chaptercolor} Jesus in the Book of Leviticus\par}
    \vspace{0.5cm}
    {\LARGE\itshape The Gospel in Symbols\par}

    \vspace{2cm}

    {\large\scshape A Comprehensive Study Guide\par}

    \vspace{1cm}

    \rule{0.6\textwidth}{0.4pt}

    \vspace{2cm}

    {\large Exploring the Christological Fulfillment\\[0.3cm]
    of Every Shadow in the Third Book of Moses\par}

    \vspace{3cm}

    {\itshape ``These are a shadow of things to come,\\
    but the substance is of Christ.''\\[0.3cm]
    --- Colossians 2:17\par}

    \vfill

    {\large Course Handbook\par}

\end{titlepage}

% ========== TABLE OF CONTENTS ==========
\frontmatter
\tableofcontents
\newpage

% ========== MAIN CONTENT ==========
\mainmatter

% ====================================================================
% CHAPTER 1: THE REVELATION OF CHRIST IN LEVITICUS
% ====================================================================
\chapter{The Revelation of Christ in Leviticus:\\The Holy God and His Way of Access}

\section{The Central Focus of the Bible}

The whole Bible has one magnificent centre --- the Person and Work of Christ. From Genesis to Revelation every line moves toward Him and finds its completion in Him.

\begin{scripture}
``You search the Scriptures, for in them you think you have eternal life; and these are they which testify of Me. But you are not willing to come to Me that you may have life.'' --- John 5:39--40
\end{scripture}

\begin{scripture}
``Beginning at Moses and all the Prophets, He expounded to them in all the Scriptures the things concerning Himself.'' --- Luke 24:27
\end{scripture}

\subsection{Veil and Vision}

When Israel read the Law, a veil lay over their hearts (2 Cor.\ 3:14--16). They saw the words but not the Word; the form but not the Face. But when the heart turns to Christ, the veil is taken away. Then everything changes:

\begin{itemize}
    \item The law becomes life.
    \item The shadow reveals its substance.
    \item The ritual unveils the Redeemer.
\end{itemize}

The same Scriptures that once concealed Him now reveal the glory of Christ, the true and living Reality. The Spirit opens the eyes of faith to see beyond the letter into the living Person --- ``The letter kills, but the Spirit gives life.'' (2 Cor.\ 3:6)

\subsection{The Structure of the Old Testament}

The Old Testament is divided into three great sections: Law, Prophets, and Psalms (Writings). Jesus declared that everything written in the Law of Moses, the Prophets, and the Psalms spoke concerning Him.

\begin{scripture}
``All things must be fulfilled which were written in the Law of Moses, and in the Prophets, and in the Psalms, concerning Me.'' --- Luke 24:44
\end{scripture}

\subsection{Christ and the Law}

\begin{scripture}
``Do not think that I came to destroy the Law or the Prophets; I did not come to destroy but to fulfill.'' --- Matthew 5:17--18
\end{scripture}

The Law revealed God's holiness and demanded righteousness. Christ fulfilled it by perfect obedience and by offering Himself as the true sacrifice. The Law demanded; Christ delivered. The Law revealed sin; Christ removed it. The Law showed the way; Christ became the Way.

\subsection{Christ in the Pentateuch}

\begin{center}
\begin{tabular}{lll}
\toprule
\textbf{Book} & \textbf{Christ Revealed As} & \textbf{Reference} \\
\midrule
Genesis & The Seed of the woman & 3:15 \\
Exodus & The Passover Lamb & 12:13 \\
Leviticus & Our High Priest and Sacrifice & Throughout \\
Numbers & The lifted Serpent & 21:8--9; John 3:14 \\
Deuteronomy & The Prophet like Moses & 18:15; Acts 3:22 \\
\bottomrule
\end{tabular}
\end{center}

\section{The Four Interpretive Principles}

To enter the riches of Leviticus, four interpretive keys must be held. God revealed truth through shadows, dual realities, divine separation, and layered meaning.

\subsection{Shadow vs.\ Substance}

Scripture calls the Law ``a shadow of good things to come, and not the very image of the things'' (Heb.\ 10:1). In Leviticus, every sacrifice, priest, and ritual was a shadow pointing to Christ, the Substance---the true and living fulfillment.

\begin{center}
\begin{tabular}{ll}
\toprule
\textbf{Shadow} & \textbf{Substance/Reality} \\
\midrule
Outline without details & Complete reality \\
Copy & Original \\
Temporary & Permanent \\
Earthly & Heavenly \\
Depends on reality & Self-existent \\
\bottomrule
\end{tabular}
\end{center}

\begin{scripture}
``These are a shadow of things to come, but the substance is of Christ.'' --- Colossians 2:17
\end{scripture}

\subsection{Reality of Duality}

Two dimensions of reality run side by side --- the visible and the invisible. The earthly tabernacle was a copy of the heavenly (Ex.\ 25:40; Heb.\ 8:5). The Levitical priests served ``the copy and shadow of the heavenly things.'' Both were real: the visible illustrated the invisible.

\subsection{Divine Separation --- The Principle of Holiness}

From the beginning God divided --- light from darkness, land from sea, Israel from the nations, Levites from the tribes, the high priest from the priests (Lev.\ 20:26). Separation is the principle of holiness.

\textbf{God's Divine Order:} Everything begins as common. From the common God divides the clean and unclean; from the clean He chooses what is holy.

\begin{center}
\textbf{Common} $\rightarrow$ \textbf{Clean} $\rightarrow$ \textbf{Holy} $\rightarrow$ \textbf{God's presence}
\end{center}

\subsection{Fourfold Meaning of Scripture (PaRDeS)}

\begin{center}
\begin{tabularx}{\textwidth}{lllX}
\toprule
\textbf{Level} & \textbf{Meaning} & \textbf{Focus} & \textbf{Description} \\
\midrule
P'shat & Literal sense & Historical & What the Word says \\
Remez & Hint/type & Prophetic & What the Word points to \\
Drash & Moral lesson & Application & What the Word teaches \\
Sod & Hidden mystery & Spiritual & What the Word reveals \\
\bottomrule
\end{tabularx}
\end{center}

\section{Understanding the Book of Leviticus}

\subsection{Name and Meaning}

\begin{itemize}
    \item \textbf{English:} Leviticus = ``pertaining to the Levites''
    \item \textbf{Hebrew:} \heb{וַיִּקְרָא} (Vayikra) = ``And He called'' (Lev.\ 1:1)
\end{itemize}

\subsection{Context and Setting}

\begin{itemize}
    \item \textbf{Location:} At the foot of Mount Sinai
    \item \textbf{Time:} About one year after the Exodus
    \item Tabernacle completed; God speaks from within it
\end{itemize}

\subsection{Exodus and Leviticus Compared}

\begin{center}
\begin{tabular}{ll}
\toprule
\textbf{Exodus} & \textbf{Leviticus} \\
\midrule
Exodus offers pardon & Leviticus offers purity \\
God's approach to man & Man's approach to God \\
Christ is Savior & Christ is Sanctifier \\
Man's guilt is prominent & Man's defilement is prominent \\
God speaks from the mount & God speaks from the tabernacle \\
Distance & Fellowship \\
Fear & Access \\
\bottomrule
\end{tabular}
\end{center}

\subsection{Structure of the Book}

\begin{center}
\begin{tabularx}{\textwidth}{llXl}
\toprule
\textbf{Chapters} & \textbf{Theme} & \textbf{Focus} & \textbf{Christ As} \\
\midrule
1--7 & Offerings & How to approach God & Our Sacrifice \\
8--10 & Priesthood & Who may approach & Our High Priest \\
11--15 & Cleanness & What defiles/purifies & Our Cleanser \\
16 & Day of Atonement & Way of access & Our Atonement \\
17--22 & Holiness & Walk and service & Our Sanctifier \\
23--25 & Feasts/Sabbaths & God's calendar & Our Fulfillment \\
26--27 & Covenant/Vows & Outcome of holiness & Our Lord and Reward \\
\bottomrule
\end{tabularx}
\end{center}

\subsection{Key Verse}

\begin{scripture}
``Be ye holy, for I am holy.'' --- Leviticus 11:44; 19:2; 20:7
\end{scripture}

\textbf{Theme:} Holiness through Atonement.

\section{God's Justice System}

Leviticus reveals that God's justice system stands upon two unchanging pillars: the Law and the Sacrifice.

\subsection{The Law --- The Holy Demand}

\begin{scripture}
``The soul that sinneth, it shall die.'' --- Ezekiel 18:4
\end{scripture}

The Law exposes sin but cannot cleanse it; it judges but cannot justify. It reveals God's absolute holiness and our desperate need for a Redeemer.

\subsection{The Sacrifice --- The Merciful Provision}

\begin{scripture}
``Without shedding of blood there is no remission.'' --- Hebrews 9:22
\end{scripture}

Where the Law condemns, the Sacrifice intercedes. The altar becomes the meeting place of justice and grace.

\subsection{The Harmony in Christ}

\begin{center}
\begin{tabular}{lll}
\toprule
\textbf{Aspect} & \textbf{Revealed in the Law} & \textbf{Fulfilled in Christ} \\
\midrule
Justice & The sinner must die & ``Christ died for our sins'' \\
Holiness & Requires obedience & ``He became obedient unto death'' \\
Mercy & Accepts a substitute & ``The Lord laid on Him our iniquity'' \\
Love & Provides a way back & ``God so loved the world'' \\
\bottomrule
\end{tabular}
\end{center}

\section{The Continuity of Redemption}

The sacrifice of Christ is not a new idea but the culmination of one eternal plan:

\begin{itemize}
    \item In Eden, God clothed Adam and Eve with skins --- innocent life for guilty life.
    \item Abel offered a lamb and found acceptance (Gen.\ 4:4).
    \item Noah built an altar after the flood (Gen.\ 8:20--21).
    \item Abraham offered Isaac but received a ram in his stead (Gen.\ 22:13).
    \item At Passover, Israel was redeemed by the blood of the lamb (Ex.\ 12:13).
    \item At Sinai, blood sealed the covenant (Ex.\ 24:8).
\end{itemize}

All these anticipated ``the Lamb slain from the foundation of the world'' (Rev.\ 13:8).


% ====================================================================
% CHAPTER 2: CHRIST IN THE OFFERINGS
% ====================================================================
\chapter{Christ in the Offerings\\(Leviticus 1--7)}

\section{The Incomprehensible Christ}

When we begin studying the offerings, we must remember that Christ Himself is infinite. His Person cannot be measured, mapped, or mastered. Paul says that even one glory of Christ --- His love --- has breadth, length, depth, and height (Eph.\ 3:18).

The five offerings are not five separate doctrines, but five windows into the same glorious Person --- each revealing one facet of His inexhaustible beauty.

\begin{scripture}
``Behold! The Lamb of God who takes away the sin of the world!'' --- John 1:29
\end{scripture}

\section{God's Order and Man's Order}

To God, everything begins with His Son. He first beholds the perfect obedience of Christ, the Finished Work offered up to Him as a sweet aroma. Therefore, in God's order, the path begins with Christ's perfection and ends in fellowship.

To man, the journey begins elsewhere. We start with our sin, guilt, and distance. Only then do we move toward the altar, toward cleansing and acceptance.

\begin{itemize}
    \item \textbf{God begins} with the Burnt Offering --- Christ's perfect devotion.
    \item \textbf{Man begins} with the Sin Offering --- our deep need.
\end{itemize}

\section{The Five Offerings Overview}

\begin{center}
\begin{tabularx}{\textwidth}{lXXX}
\toprule
\textbf{Offering} & \textbf{Hebrew} & \textbf{Purpose} & \textbf{Christ Revealed} \\
\midrule
Burnt Offering & \heb{עֹלָה} (\textit{'olah}) & Total consecration & Complete devotion to God \\
Grain Offering & \heb{מִנְחָה} (\textit{minhah}) & Thanksgiving & Perfect humanity \\
Peace Offering & \heb{שְׁלָמִים} (\textit{shelamim}) & Fellowship & Our peace with God \\
Sin Offering & \heb{חַטָּאת} (\textit{hattat}) & Atonement for sin & Bearer of our sin \\
Trespass Offering & \heb{אָשָׁם} (\textit{'asham}) & Restitution & Restorer of all \\
\bottomrule
\end{tabularx}
\end{center}

\section{Key Truths in the Offerings}

\begin{center}
\begin{tabularx}{\textwidth}{lXX}
\toprule
\textbf{Truth} & \textbf{Meaning in Leviticus} & \textbf{Fulfilled in Christ} \\
\midrule
God provides the sacrifice & God chose the offering & Christ is God's Lamb \\
Fire shows judgment & Fire never went out & Christ bore full judgment \\
Fat belongs only to God & Richest part for God & Christ's heart delighted God \\
Blood meets God first & Applied first to altar & Cross satisfied God first \\
No leaven & No corruption allowed & Christ is sinless \\
Salt = covenant & Salt in every offering & Christ's covenant eternal \\
Ashes = finished work & Nothing left to burn & ``It is finished'' \\
\bottomrule
\end{tabularx}
\end{center}

\section{Outside the Camp}

Leviticus uses two different Hebrew verbs for ``burn,'' each carrying a distinct spiritual meaning:

\begin{center}
\begin{tabular}{lllll}
\toprule
\textbf{Verb} & \textbf{Meaning} & \textbf{Location} & \textbf{Symbolism} & \textbf{Fulfillment} \\
\midrule
\heb{קָטַר} (\textit{qatar}) & To rise as incense & On the altar & Acceptance & Eph.\ 5:2 \\
\heb{שָׂרַף} (\textit{saraph}) & To burn in judgment & Outside camp & Curse-bearing & Heb.\ 13:12 \\
\bottomrule
\end{tabular}
\end{center}

At Calvary, the fragrance of His obedience and the fire of judgment met in one place.


% ====================================================================
% CHAPTER 3: CHRIST IN THE PRIESTHOOD
% ====================================================================
\chapter{Christ in the Priesthood\\(Leviticus 8--10)}

\begin{scripture}
``We have such a High Priest, who is seated at the right hand of the throne of the Majesty in the heavens.'' --- Hebrews 8:1
\end{scripture}

Aaron's priesthood was temporary and symbolic. Christ's priesthood is eternal and real --- the fulfillment of all Levitical shadows.

\section{The Ten Steps of Priestly Consecration}

\begin{center}
\begin{tabularx}{\textwidth}{lXX}
\toprule
\textbf{Step} & \textbf{Leviticus 8} & \textbf{Fulfilled in Christ} \\
\midrule
1. Selection & Chosen by God (v.2) & ``You are My Son... a priest forever'' (Heb.\ 5:5--6) \\
2. Approach & Brought Near (v.6) & Christ approached by His own blood \\
3. Washing & Purity (v.6) & Christ fulfilled all righteousness \\
4. Clothing & Righteousness (v.7) & Clothed in humanity and glory \\
5. Anointing & Spirit's Power (v.12) & Anointed with the Spirit (Luke 4:18) \\
6. Substitution & Sin Offering (v.14) & Became the sin offering (1 Pet.\ 2:24) \\
7. Consecration & Dedication (v.23,30) & Perfect obedience \\
8. Service & Ministry (v.27) & ``My food is to do His will'' \\
9. Separation & Sanctification (v.30) & ``For their sakes I sanctify Myself'' \\
10. Participation & Communion (v.31) & ``He who eats My flesh abides in Me'' \\
\bottomrule
\end{tabularx}
\end{center}

\section{The Superiority of Christ's Priesthood}

\begin{enumerate}
    \item \textbf{Superior in Person} --- He is sinless and divine (Heb.\ 7:26--28)
    \item \textbf{Superior in Order} --- ``After the order of Melchizedek'' (Heb.\ 7:17)
    \item \textbf{Superior in Appointment} --- By divine oath, not law (Heb.\ 7:21)
    \item \textbf{Superior in Sympathy} --- He understands our weakness (Heb.\ 4:15)
\end{enumerate}

\section{Comparison: Levitical vs.\ Christ's Priesthood}

\begin{center}
\begin{tabular}{lll}
\toprule
\textbf{Aspect} & \textbf{Levitical Priesthood} & \textbf{Christ's Priesthood} \\
\midrule
Order & Aaronic (temporary) & Melchizedek (eternal) \\
Appointment & By law & By divine oath \\
Character & Sinful men & Sinless Son of God \\
Sacrifice & Many, repeated & One, perfect, final \\
Sanctuary & Earthly & Heavenly \\
Covenant & Old, fading & New, everlasting \\
Effect & Could not perfect & Perfected forever \\
Duration & Ends with death & Lives forever \\
Access & Limited & Open to all believers \\
\bottomrule
\end{tabular}
\end{center}

\section{The Three Appearances (Leviticus 9 and Hebrews 9)}

\begin{center}
\begin{tabularx}{\textwidth}{llXX}
\toprule
\textbf{Time} & \textbf{Leviticus 9} & \textbf{Christ's Appearance} & \textbf{Hebrews 9} \\
\midrule
Past & At the Altar (v.7) & He appeared to put away sin & 9:26 \\
Present & In the Tabernacle (v.23a) & He appears now for us & 9:24 \\
Future & Before the People (v.23b) & He shall appear apart from sin & 9:28 \\
\bottomrule
\end{tabularx}
\end{center}

\textbf{The Three Eternals:}
\begin{itemize}
    \item \textbf{Eternal Spirit} (Heb.\ 9:14) --- Power behind His sacrifice
    \item \textbf{Eternal Redemption} (Heb.\ 9:12) --- Result of His heavenly ministry
    \item \textbf{Eternal Inheritance} (Heb.\ 9:15) --- Outcome of His return
\end{itemize}


% ====================================================================
% CHAPTER 4: THE DAY OF ATONEMENT
% ====================================================================
\chapter{Christ in the Day of Atonement\\(Leviticus 16)}

\textbf{Theme:} The once-a-year atonement of Israel is fulfilled in the once-for-all atonement of Christ.

\section{The Pinnacle of Leviticus}

Among all the chapters of the Pentateuch, one chapter rises like the pinnacle of a holy mountain --- Leviticus 16, the Day of Atonement.

If the book of Leviticus is a sanctuary, then chapter 16 is the Holy of Holies.

\section{The Meaning of Atonement}

The Hebrew word \heb{כִּפֵּר} (\textit{kaphar}) means:
\begin{itemize}
    \item ``to cover''
    \item ``to make reconciliation''
    \item ``to make satisfaction''
\end{itemize}

\textbf{Distinction:}
\begin{itemize}
    \item \textbf{Old Covenant:} Sin covered until Christ (Rom.\ 3:25)
    \item \textbf{New Covenant:} Sin taken away forever (John 1:29)
\end{itemize}

\section{The Two Goats: Dual Aspect of Atonement}

Two goats were taken for one sin offering (16:5) --- not two sacrifices but two halves of one act.

\begin{center}
\begin{tabularx}{\textwidth}{lXXX}
\toprule
\textbf{Goat} & \textbf{Action} & \textbf{Meaning} & \textbf{Fulfillment} \\
\midrule
Slain goat & Blood sprinkled on mercy seat & Propitiation --- satisfaction of God's justice & Rom.\ 3:25 \\
Scapegoat (\heb{עֲזָאזֵל}) & Bearing sins into wilderness & Expiation --- removal of sin & John 1:29 \\
\bottomrule
\end{tabularx}
\end{center}

One goat represents Godward satisfaction; the other, manward removal --- both fulfilled in the Cross.

\section{Key Actions on the Day}

\begin{enumerate}
    \item \textbf{Preparation} (vv.\ 3--4): Aaron washed and put on plain linen garments
    \item \textbf{Personal Sin Offering} (v.\ 6): Bull for himself and his household
    \item \textbf{The Slain Goat} (vv.\ 15--19): Blood sprinkled seven times on mercy seat
    \item \textbf{Entrance within the Veil} (vv.\ 11--14): With blood and incense
    \item \textbf{The Scapegoat Sent Away} (vv.\ 20--22): Bearing confessed sins
    \item \textbf{Burning Outside the Camp} (v.\ 27): Complete consumption
    \item \textbf{The Sabbath of Rest} (vv.\ 29--31): No work
\end{enumerate}

\section{The Annual vs.\ the Eternal}

\begin{center}
\begin{tabular}{lll}
\toprule
\textbf{Contrast} & \textbf{Day of Atonement} & \textbf{Christ's Atonement} \\
\midrule
Frequency & Every year & Once for all (Heb.\ 9:26) \\
Priest & Many, mortal & One, eternal \\
Blood & Of animals & Of Christ \\
Sanctuary & Earthly, symbolic & Heavenly, real \\
Result & Temporary covering & Eternal redemption \\
Access & Restricted & Open to all believers \\
Memory of sin & Continual & Sin remembered no more \\
\bottomrule
\end{tabular}
\end{center}

\section{The Mercy Seat and the Cross}

The \heb{כַּפֹּרֶת} (\textit{kapporet}, mercy seat) was the golden lid covering the ark of the covenant. Beneath lay the tablets of the law --- God's righteous standard; above hovered His glory. Only when blood was sprinkled did judgment and mercy meet.

\begin{scripture}
``Righteousness and peace have kissed each other.'' --- Psalm 85:10
\end{scripture}

At the Cross, the blood of Christ satisfied the law beneath and revealed the glory above. The place of judgment became the throne of grace.

\section{Doctrinal Synthesis}

\begin{enumerate}
    \item \textbf{Substitution} --- The innocent died for the guilty
    \item \textbf{Identification} --- The sins were confessed and transferred
    \item \textbf{Propitiation} --- God's holiness satisfied by blood on the mercy seat
    \item \textbf{Expiation} --- Sin removed and remembered no more
    \item \textbf{Access} --- The veil opened; believers enter boldly (Heb.\ 10:19--22)
    \item \textbf{Intercession} --- Christ now appears for us continually
    \item \textbf{Rest} --- The finished work invites faith, not labor
\end{enumerate}

\begin{scripture}
``By one offering He has perfected forever those who are being sanctified.'' --- Hebrews 10:14
\end{scripture}


% ====================================================================
% CHAPTER 5: LAWS OF CLEANSING AND HOLINESS
% ====================================================================
\chapter{Christ in the Laws of Cleansing and Holiness\\(Leviticus 11--15; 17--20)}

\textbf{Theme:} ``Be holy, for I am holy'' --- Cleansing and sanctification fulfilled in Christ.

\section{Introduction: The God Who Separates and Sanctifies}

After the atonement of Leviticus 16, the next great section unfolds the principle of holiness --- what it means to live as a cleansed and separated people.

\begin{scripture}
``For I am the LORD your God: you shall therefore sanctify yourselves, and you shall be holy; for I am holy.'' --- Leviticus 11:44
\end{scripture}

Atonement makes access possible; holiness makes fellowship continual.

\section{The Vocabulary of Purity}

\begin{center}
\begin{tabular}{lllll}
\toprule
\textbf{Hebrew} & \textbf{Transliteration} & \textbf{Meaning} & \textbf{Reference} & \textbf{Fulfillment} \\
\midrule
\heb{טָהוֹר} & \textit{tahor} & Clean, pure & Lev.\ 11:47 & Matt.\ 5:8 \\
\heb{טָמֵא} & \textit{tame'} & Unclean & Lev.\ 11:47 & Mark 1:40--45 \\
\heb{קָדַשׁ} & \textit{qadash} & To sanctify & Lev.\ 20:7 & John 17:19 \\
\heb{נָזָה} & \textit{nazah} & To sprinkle & Lev.\ 14:7 & Heb.\ 12:24 \\
\heb{דָּם} & \textit{dam} & Blood & Lev.\ 17:11 & Heb.\ 9:14 \\
\bottomrule
\end{tabular}
\end{center}

\section{Laws of Cleanness (Leviticus 11--15)}

\subsection{Clean and Unclean Animals (Chapter 11)}

\begin{center}
\begin{tabularx}{\textwidth}{lXXX}
\toprule
\textbf{Category} & \textbf{Criterion} & \textbf{Symbolism} & \textbf{Fulfillment} \\
\midrule
Land animals & Split hoof + chew cud & Discernment + meditation & Walk and Word \\
Sea creatures & Fins and scales & Movement and protection & Separation \\
Birds & Avoid scavengers & Reject defilement & Purity of mind \\
\bottomrule
\end{tabularx}
\end{center}

In Mark 7:18--19, Jesus declared all foods clean, showing that external distinctions had served their symbolic purpose. True cleanness comes from the heart purified by faith (Acts 15:9).

\subsection{Laws of Leprosy (Chapters 13--14)}

Leprosy in Scripture is a vivid picture of sin:
\begin{itemize}
    \item It begins unseen
    \item It spreads silently
    \item It defiles completely
    \item It separates from fellowship
\end{itemize}

\textbf{Fulfillment:} Christ cleanses the leper by touch (Mark 1:40--45), signifying that holiness is contagious in Him, not defilement.

\section{The Sanctity of Blood (Leviticus 17)}

\begin{scripture}
``For the life of the flesh is in the blood, and I have given it to you upon the altar to make atonement for your souls.'' --- Leviticus 17:11
\end{scripture}

This verse is the theological center of Leviticus. Life belongs to God, and blood --- the carrier of life --- must be treated as sacred.

\section{Moral and Ethical Holiness (Leviticus 18--20)}

Chapters 18--20 move from ceremonial purity to moral purity --- from the shadow to the substance of holiness.

\begin{scripture}
``You shall be holy to Me, for I the LORD am holy, and have separated you from the peoples, that you should be Mine.'' --- Leviticus 20:26
\end{scripture}

\section{Christ the Fulfillment of Cleansing}

\begin{center}
\begin{tabular}{lll}
\toprule
\textbf{Symbol} & \textbf{Meaning} & \textbf{Fulfillment} \\
\midrule
Water & Cleansing & ``Born of water and Spirit'' (John 3:5) \\
Blood & Atonement & ``The blood of Jesus cleanses'' (1 John 1:7) \\
Oil & Consecration & ``Anointed with the Spirit'' (Acts 10:38) \\
Fire & Purification & ``Baptize with Spirit and fire'' (Matt.\ 3:11) \\
\bottomrule
\end{tabular}
\end{center}

Cleansing in Leviticus was ritual and repeated; in Christ, it is spiritual and complete.


% ====================================================================
% CHAPTER 6: THE FEASTS OF THE LORD
% ====================================================================
\chapter{Christ in the Feasts of the Lord\\(Leviticus 23)}

\textbf{Theme:} The Prophetic Calendar of Redemption --- Christ in Every Appointed Time.

\section{God's Calendar of Redemption}

Leviticus 23 presents the Feasts of the Lord --- not Israel's feasts, but Yahweh's appointed times (\heb{מוֹעֲדִים}, \textit{mo'adim}). They are the divine blueprint of God's redemptive history.

\begin{scripture}
``These are the appointed feasts of the LORD, holy convocations, which you shall proclaim in their seasons.'' --- Leviticus 23:4
\end{scripture}

\section{The Seven Feasts}

\begin{center}
\begin{tabularx}{\textwidth}{lllX}
\toprule
\textbf{Feast} & \textbf{Hebrew} & \textbf{Date} & \textbf{Fulfillment in Christ} \\
\midrule
1. Sabbath & \heb{שַׁבָּת} & Weekly & Eternal rest in Christ \\
2. Passover & \heb{פֶּסַח} & Nisan 14 & Death of Christ \\
3. Unleavened Bread & \heb{מַצּוֹת} & Nisan 15--21 & Sinless burial \\
4. Firstfruits & \heb{בִּכּוּרִים} & Nisan 17 & Resurrection \\
5. Pentecost & \heb{שָׁבוּעוֹת} & Sivan 6 & Spirit's outpouring \\
6. Trumpets & \heb{תְּרוּעָה} & Tishri 1 & Christ's return \\
7. Atonement & \heb{כִּפֻּרִים} & Tishri 10 & Israel's repentance \\
8. Tabernacles & \heb{סֻכּוֹת} & Tishri 15--22 & Kingdom fullness \\
\bottomrule
\end{tabularx}
\end{center}

The first four feasts were fulfilled in Christ's first coming; the last three will be fulfilled in His second coming.

\section{The Spring Feasts (Fulfilled)}

\subsection{Passover (Leviticus 23:4--5)}

\begin{scripture}
``When I see the blood, I will pass over you.'' --- Exodus 12:13
\end{scripture}

\begin{center}
\begin{tabular}{lll}
\toprule
\textbf{Element} & \textbf{Symbolism} & \textbf{Fulfillment} \\
\midrule
Lamb without blemish & Innocence & ``Behold the Lamb of God'' \\
Blood on doorposts & Substitution & ``Without shedding of blood...'' \\
No bone broken & Integrity & John 19:36 \\
\bottomrule
\end{tabular}
\end{center}

\subsection{Unleavened Bread (Leviticus 23:6--8)}

No leaven = Absence of corruption. Christ's body ``saw no corruption'' (Acts 2:27).

\subsection{Firstfruits (Leviticus 23:9--14)}

Held on the very morning of Christ's resurrection.

\begin{scripture}
``Now is Christ risen from the dead, and become the firstfruits of them that slept.'' --- 1 Corinthians 15:20
\end{scripture}

\subsection{Pentecost (Leviticus 23:15--22)}

Fifty days after Firstfruits. Two leavened loaves = Jew and Gentile united in one body.

\section{The Fall Feasts (Yet to be Fulfilled)}

\subsection{Feast of Trumpets (Leviticus 23:23--25)}

The trumpet blast symbolizes:
\begin{itemize}
    \item Call to assemble --- Rapture of the Church (1 Thess.\ 4:16--17)
    \item Awakening --- Revival and repentance
    \item New beginning --- Restoration of God's people
\end{itemize}

\subsection{Day of Atonement (Leviticus 23:26--32)}

\begin{scripture}
``They shall look on Him whom they pierced.'' --- Zechariah 12:10
\end{scripture}

\subsection{Feast of Tabernacles (Leviticus 23:33--44)}

\begin{scripture}
``The Word became flesh and tabernacled among us.'' --- John 1:14
\end{scripture}

\begin{scripture}
``Behold, the tabernacle of God is with men.'' --- Revelation 21:3
\end{scripture}

\section{The Theology of the Feasts}

\begin{enumerate}
    \item \textbf{Redemption} (Passover) --- The Lamb slain
    \item \textbf{Separation} (Unleavened Bread) --- The old life removed
    \item \textbf{Resurrection} (Firstfruits) --- New life begun
    \item \textbf{Empowerment} (Pentecost) --- Spirit poured out
    \item \textbf{Regathering} (Trumpets) --- Saints summoned
    \item \textbf{Reconciliation} (Atonement) --- Israel restored
    \item \textbf{Rejoicing} (Tabernacles) --- God dwelling forever
\end{enumerate}


% ====================================================================
% CHAPTER 7: COVENANT, SABBATH, AND JUBILEE
% ====================================================================
\chapter{Christ in the Covenant, the Sabbath, and the Jubilee\\(Leviticus 25--26)}

\textbf{Theme:} Rest, Redemption, and Restoration --- Christ the Lord of the Sabbath and the Proclaimer of Jubilee.

\section{The Rhythm of Divine Rest}

Leviticus 25--26 concludes the covenantal section by revealing three great theological realities:

\begin{enumerate}
    \item \textbf{Rest} (Sabbath and Sabbatical Year) --- rest from labor and trust in divine provision
    \item \textbf{Redemption} (Jubilee) --- liberty and restoration by grace
    \item \textbf{Relationship} (Covenant Blessings) --- fellowship with God
\end{enumerate}

\begin{scripture}
``The land is Mine; for you are strangers and sojourners with Me.'' --- Leviticus 25:23
\end{scripture}

\section{Hebrew Word Studies}

\begin{center}
\begin{tabular}{lllll}
\toprule
\textbf{Hebrew} & \textbf{Transliteration} & \textbf{Meaning} & \textbf{Reference} & \textbf{Fulfillment} \\
\midrule
\heb{שַׁבָּת} & \textit{shabbat} & Rest & Lev.\ 25:2 & Matt.\ 11:28 \\
\heb{יוֹבֵל} & \textit{yovel} & Jubilee, liberty & Lev.\ 25:10 & Luke 4:18 \\
\heb{גָּאַל} & \textit{ga'al} & To redeem & Lev.\ 25:25 & Eph.\ 1:7 \\
\heb{בְּרִית} & \textit{berit} & Covenant & Lev.\ 26:9 & Heb.\ 8:6 \\
\heb{חֶסֶד} & \textit{hesed} & Steadfast love & Lev.\ 26:42 & John 1:17 \\
\bottomrule
\end{tabular}
\end{center}

\section{The Year of Jubilee}

\begin{scripture}
``You shall hallow the fiftieth year, and proclaim liberty throughout all the land to all its inhabitants.'' --- Leviticus 25:10
\end{scripture}

\subsection{The Threefold Liberation}

\begin{center}
\begin{tabular}{lll}
\toprule
\textbf{Area} & \textbf{Action} & \textbf{Spiritual Fulfillment} \\
\midrule
Debt & All debts cancelled & Forgiveness through Christ's blood \\
Slavery & All slaves released & Deliverance from sin's bondage \\
Inheritance & Land restored & Restoration of spiritual inheritance \\
\bottomrule
\end{tabular}
\end{center}

The Jubilee trumpet blew on the Day of Atonement --- liberty rooted in atonement. True freedom flows from forgiveness.

\section{The Kinsman-Redeemer}

The laws of redemption introduce one of the most beautiful titles of Christ --- the \heb{גֹּאֵל} (\textit{Go'el}, Redeemer).

\begin{scripture}
``I know that my Redeemer lives.'' --- Job 19:25
\end{scripture}

\begin{center}
\begin{tabularx}{\textwidth}{lXX}
\toprule
\textbf{Provision} & \textbf{Description} & \textbf{Fulfillment in Christ} \\
\midrule
Redemption of property (25:25) & Near relative buys back lost land & Christ restores lost inheritance \\
Redemption of persons (25:47--49) & Kinsman redeems enslaved family & Christ redeems us from slavery \\
Relationship required & Only a near kinsman qualified & Christ became man to qualify \\
\bottomrule
\end{tabularx}
\end{center}

\section{Sabbath Year vs.\ Jubilee Year}

\begin{center}
\begin{tabular}{lll}
\toprule
\textbf{Feature} & \textbf{Sabbath Year} & \textbf{Jubilee Year} \\
\midrule
Frequency & Every 7th year & Every 50th year \\
Land & Rests & Returns to original owner \\
Debts & Suspended & Cancelled \\
Servants & Rest & Released \\
Fulfillment & Spiritual rest (Matt.\ 11:28) & Full redemption (Luke 4:18--19) \\
\bottomrule
\end{tabular}
\end{center}

\section{The Covenant Blessings and Curses (Leviticus 26)}

\subsection{Blessings for Obedience (26:1--13)}

\begin{scripture}
``I will walk among you, and be your God, and you shall be My people.'' --- Leviticus 26:12
\end{scripture}

\subsection{Discipline for Disobedience (26:14--39)}

Even in judgment, mercy prevailed:

\begin{scripture}
``Yet for all that... I will not cast them away... I will remember My covenant.'' --- Leviticus 26:44--45
\end{scripture}

\section{Christ's Fulfillment}

\begin{scripture}
``The Spirit of the Lord is upon Me... to preach deliverance to the captives, to proclaim the acceptable year of the Lord.'' --- Luke 4:18--19
\end{scripture}

Christ is our eternal Jubilee. He releases us from sin's slavery, restores our lost inheritance, and reconciles us to the Father.


% ====================================================================
% CHAPTER 8: DEDICATION, VOWS, AND THE PRESENCE OF GOD
% ====================================================================
\chapter{Christ in Dedication, Vows, and the Presence of God\\(Leviticus 27)}

\textbf{Theme:} Holiness Expressed in Devotion --- Christ the Perfectly Consecrated One.

\section{From Atonement to Dedication}

Leviticus ends not with the sacrifices, feasts, or jubilees, but with vows and dedications --- the voluntary acts of consecration.

\begin{scripture}
``Behold, I come to do Your will, O God.'' --- Hebrews 10:7
\end{scripture}

Having revealed how sinners draw near through blood, God now shows how the redeemed respond through devotion. Atonement removes the barrier; holiness builds the relationship; dedication expresses love's response.

\section{Structure of Leviticus 27}

\begin{center}
\begin{tabular}{llc}
\toprule
\textbf{Section} & \textbf{Topic} & \textbf{Verses} \\
\midrule
I & Valuation of persons vowed to the Lord & 1--8 \\
II & Dedication of animals & 9--13 \\
III & Dedication of houses & 14--15 \\
IV & Dedication of fields & 16--25 \\
V & Devoted (\heb{חֵרֶם}) things --- irrevocable vows & 26--29 \\
VI & The law of the tithe & 30--33 \\
VII & Summary & 34 \\
\bottomrule
\end{tabular}
\end{center}

\section{Key Hebrew Terms}

\begin{center}
\begin{tabular}{lllll}
\toprule
\textbf{Hebrew} & \textbf{Transliteration} & \textbf{Meaning} & \textbf{Reference} & \textbf{Fulfillment} \\
\midrule
\heb{נֶדֶר} & \textit{neder} & Vow, promise & Lev.\ 27:2 & Heb.\ 10:7 \\
\heb{חֵרֶם} & \textit{herem} & Devoted thing & Lev.\ 27:28 & John 17:19 \\
\heb{מַעֲשֵׂר} & \textit{ma'aser} & Tithe & Lev.\ 27:30 & 1 Cor.\ 15:20 \\
\heb{קֹדֶשׁ} & \textit{qodesh} & Holy & Lev.\ 27:30 & Luke 1:35 \\
\bottomrule
\end{tabular}
\end{center}

\section{Devoted Things --- the \heb{חֵרֶם}}

\begin{scripture}
``Nothing that a man devotes to the LORD, whether man or beast or field, shall be sold or redeemed; every devoted thing is most holy to the LORD.'' --- Leviticus 27:28
\end{scripture}

The \heb{חֵרֶם} referred to something irreversibly set apart --- either for destruction (in judgment) or for God's exclusive use. All \heb{חֵרֶם} points to the absolute claims of holiness.

\begin{scripture}
``For their sakes I sanctify Myself.'' --- John 17:19
\end{scripture}

\section{The Law of the Tithe}

\begin{scripture}
``All the tithe of the land, whether seed of the land or fruit of the tree, is the LORD's; it is holy to the LORD.'' --- Leviticus 27:30
\end{scripture}

Tithing recognized God as owner of all things. It was not merely an economic rule but an act of worship --- the giving of the first and best.

\section{The Theology of Dedication}

\begin{enumerate}
    \item \textbf{Dedication is voluntary but sacred.} The vow was optional, but once made, it was binding.
    \item \textbf{Dedication involves valuation and redemption.} Whatever is offered must be measured; consecration has a cost.
    \item \textbf{Dedication culminates in Christ.} He is the true Nazarite of God --- vowed, consecrated, and crowned with holiness.
    \item \textbf{Dedication flows from gratitude.} Only those redeemed and cleansed can dedicate themselves to God.
\end{enumerate}

\begin{scripture}
``I beseech you therefore, brethren, by the mercies of God, to present your bodies a living sacrifice.'' --- Romans 12:1
\end{scripture}

\section{Doctrinal Synthesis}

\begin{enumerate}
    \item \textbf{Dedication completes redemption.} Salvation received must become service returned.
    \item \textbf{Christ embodies total consecration.} Every vow finds its perfection in Him.
    \item \textbf{The Spirit applies this holiness to believers.} Our bodies and possessions belong to God.
    \item \textbf{The Church lives as a devoted people.} ``You are a chosen generation, a royal priesthood'' (1 Pet.\ 2:9).
    \item \textbf{All ends with holiness.} The last word of Leviticus is \heb{קֹדֶשׁ} --- holy.
\end{enumerate}

\section{Closing Meditation}

\begin{scripture}
``Holy to the LORD.'' --- Leviticus 27:30
\end{scripture}

\begin{scripture}
``For their sakes I sanctify Myself.'' --- John 17:19
\end{scripture}

Leviticus begins with blood on the altar and ends with holiness to the Lord. The journey from guilt to glory, from offering to devotion, is complete in Christ.

He is the One who vowed, offered, redeemed, and devoted all to the Father --- and in Him, we too are called ``holy unto the Lord.''

\begin{scripture}
``The grace of God that brings salvation has appeared... teaching us to live soberly, righteously, and godly in this present world.'' --- Titus 2:11--12
\end{scripture}


% ========== BACK MATTER ==========
\backmatter

\chapter*{Conclusion}

The book of Leviticus is not a dry collection of ancient rituals but a living revelation of Jesus Christ. From the first sacrifice to the final vow, every shadow finds its substance in Him.

\begin{center}
\begin{tabularx}{\textwidth}{lX}
\toprule
\textbf{Levitical Theme} & \textbf{Christ the Fulfillment} \\
\midrule
The Offerings & The perfect Sacrifice \\
The Priesthood & The eternal High Priest \\
The Day of Atonement & The once-for-all Redeemer \\
The Laws of Cleanness & The Holy Sanctifier \\
The Feasts & The Appointed Time \\
The Sabbath and Jubilee & The Rest-Giver and Liberator \\
Vows and Dedication & The Consecrated One \\
\bottomrule
\end{tabularx}
\end{center}

\begin{scripture}
``These are a shadow of things to come, but the substance is of Christ.'' --- Colossians 2:17
\end{scripture}

\vspace{1cm}

\begin{center}
\textit{Soli Deo Gloria}
\end{center}

\end{document}
