\chapter{The Revelation of Christ in Leviticus}

\section{The Central Focus of the Bible --- Christ the Fulfillment of all Scripture}

The whole Bible has one magnificent centre --- the Person and Work of Christ. From Genesis to Revelation every line moves toward Him and finds its completion in Him.

\begin{scripture}
``You search the Scriptures, for in them you think you have eternal life; and these are they which testify of Me. But you are not willing to come to Me that you may have life.''
\scriptureref{John 5:39--40}
\end{scripture}

\begin{scripture}
``Beginning at Moses and all the Prophets, He expounded to them in all the Scriptures the things concerning Himself.''
\scriptureref{Luke 24:27}
\end{scripture}

\subsection{Veil and Vision}

When Israel read the Law, a veil lay over their hearts (2 Corinthians 3:14--16). They saw the words but not the Word; the form but not the Face. But when the heart turns to Christ, the veil is taken away. Then everything changes:

\begin{itemize}
\item The law becomes life.
\item The shadow reveals its substance.
\item The ritual unveils the Redeemer.
\end{itemize}

The same Scriptures that once concealed Him now reveal the glory of Christ, the true and living Reality. The Spirit opens the eyes of faith to see beyond the letter into the living Person --- \textit{``The letter kills, but the Spirit gives life.''} (2 Corinthians 3:6)

\subsection{The Structure of the Old Testament}

The Old Testament is divided into three great sections: Law, Prophets, and Psalms (Writings). Jesus declared that everything written in the Law of Moses, the Prophets, and the Psalms spoke concerning Him.

\begin{scripture}
``All things must be fulfilled which were written in the Law of Moses, and in the Prophets, and in the Psalms, concerning Me.''
\scriptureref{Luke 24:44}
\end{scripture}

\subsection{Christ and the Law}

\begin{scripture}
``Do not think that I came to destroy the Law or the Prophets; I did not come to destroy but to fulfill.''
\scriptureref{Matthew 5:17--18}
\end{scripture}

The Law revealed God's holiness and demanded righteousness. Christ fulfilled it by perfect obedience and by offering Himself as the true sacrifice. The Law demanded; Christ delivered. The Law revealed sin; Christ removed it. The Law showed the way; Christ became the Way.

\subsection{Christ and the Prophets}

\textit{``The Law and the Prophets were until John; since that time the kingdom of God is preached''} (Luke 16:16). \textit{``To Him give all the prophets witness''} (Acts 10:43). Every prophet pointed to Him; Christ came as their fulfillment.

\subsection{Christ and the Psalms}

\textit{``All things must be fulfilled \ldots in the Psalms concerning Me''} (Luke 24:44). The Psalms unveil His suffering (Psalm 22), resurrection (Psalm 16), kingship (Psalm 2), and priesthood (Psalm 110). What the Law required, what the Prophets foretold, and what the Psalms expressed---Christ fulfilled completely.

\subsection{Christ Revealed to the Disciples}

\textit{``Did not our hearts burn within us while He talked with us on the road?''} (Luke 24:32). Every sacrifice, law, song, and prophecy finds meaning only in Christ Himself.

\subsection{Apostolic Confirmation}

\begin{tabularx}{\textwidth}{lX}
\toprule
\textbf{Reference} & \textbf{Witness of Fulfillment} \\
\midrule
Romans 3:21 & \textit{``The righteousness of God \ldots witnessed by the Law and the Prophets.''} \\
Hebrews 10:7 & \textit{``Behold, I have come---In the volume of the book it is written of Me---To do Your will, O God.''} \\
Acts 26:22--23 & Paul preached nothing beyond what the Prophets and Moses said. \\
John 5:46 & \textit{``If you believed Moses, you would believe Me. for he wrote about Me''} \\
\bottomrule
\end{tabularx}

\subsection{Summary}

The Law shows the holiness of God. The Prophets reveal the purpose of God. The Psalms express the heart of God. All three converge in Christ, the fullness of God.

\begin{itemize}
\item In Genesis He is the Seed of the woman (3:15).
\item In Exodus the Passover Lamb (12:13).
\item In Leviticus our High Priest and Sacrifice.
\item In Numbers the lifted Serpent (21:8--9 cf. John 3:14).
\item In Deuteronomy the Prophet like Moses (18:15 cf. Acts 3:22).
\end{itemize}

\section{Christology and Revelation --- Eagle and Keyhole View}

``Christology'' means the study of the Person and Work of Christ. The Bible may be examined through a keyhole --- individual doctrines or events --- or viewed from the eagle's height, perceiving its grand unity in the eternal purpose of God. Without knowing Christ, one cannot truly know God, worship Him, or be transformed by Him. The Bible does not simply reveal facts \textbf{about} Christ --- it reveals \textbf{Christ Himself}. God gave us the Scriptures not only to inform our minds but to \textbf{transform our hearts}. The purpose of revelation is \textbf{relationship}. The goal of doctrine is \textbf{devotion}. Unless we \textbf{know Him}, we cannot be changed by Him. Unless we \textbf{see Him}, we cannot follow Him. Unless we \textbf{abide in Him}, we cannot obey or worship Him.

\begin{scripture}
``And this is life eternal, that they might know Thee, the only true God, and Jesus Christ, whom Thou hast sent''
\scriptureref{John 17:3}
\end{scripture}

Knowledge without communion becomes pride. Doctrine without devotion becomes dryness. But when the truth of Christ fills the heart, it produces worship, obedience, and holiness.

Paul said, \textit{``That I may know Him, and the power of His resurrection, and the fellowship of His sufferings.''} (Philippians 3:10)

Every book of the Bible --- including Leviticus --- is meant to lead us into that kind of \textbf{knowing}. When we study Leviticus, we are not just studying ancient rituals; we are learning what it means to \textbf{draw near to God} through the sacrifice of His Son. The revelation of Christ is the root of all transformation. The more clearly we see Him, the more deeply we are changed into His image.

\textit{``We all, with unveiled face beholding as in a mirror the glory of the Lord, are transformed into the same image''} (2 Corinthians 3:18). Therefore, when we open this book, let us not seek knowledge alone --- let us seek \textbf{Him}. For only those who know Him can truly follow, obey, and worship Him.

Every true revelation of God produces humility. When \textbf{Isaiah} saw the Lord high and lifted up, he cried, \textit{``Woe is me! for I am undone; because I am a man of unclean lips.''} (Isaiah 6:5)

When \textbf{Peter} saw the miraculous power of Jesus, he fell at His knees and said, \textit{``Depart from me, for I am a sinful man, O Lord.''} (Luke 5:8)

When \textbf{John} saw the risen Christ in His glory, he said, \textit{``When I saw Him, I fell at His feet as dead.''} (Revelation 1:17)

The closer we come to the Light, the more we see our own unworthiness. Yet in that very moment of brokenness, \textbf{grace lifts us up}. Isaiah was cleansed by a live coal from the altar. Peter was called to become a fisher of men. John was commissioned to write the Revelation. The same Lord who humbles also restores. To \textbf{see Him} truly is to be \textbf{changed} forever.

\section{The Four Interpretive Principles of Revelation in Leviticus}

To enter the riches of Leviticus, four interpretive keys must be held. God revealed truth through \textbf{shadows}, \textbf{dual realities}, \textbf{divine separation}, and \textbf{layered meaning.}

\subsection{Shadow vs Substance}

Scripture calls the Law \textit{``a shadow of good things to come, and not the very image of the things''} (Hebrews 10:1). In Leviticus, every sacrifice, priest, and ritual was a shadow pointing to Christ, the Substance---the true and living fulfillment. --- Christ.

A shadow is real but incomplete; it depends on the substance for its form.

\begin{tabularx}{\textwidth}{lX}
\toprule
\textbf{Shadow} & \textbf{Substance / Reality} \\
\midrule
Outline without details & Complete reality \\
Copy & Original \\
Temporary & Permanent \\
Earthly & Heavenly \\
Depends on reality & Self-existent \\
\bottomrule
\end{tabularx}

\textit{``These are a shadow of things to come, but the substance is of Christ.''} (Colossians 2:17)

\textit{``The law having a shadow of the good things to come can never make those who approach perfect.''} (Hebrews 10:1).

The shadow \textbf{teaches}; the substance \textbf{fulfills}.

The shadow \textbf{points}; the substance \textbf{provides}.

The shadow \textbf{shows the form}; the substance \textbf{gives the life}.

\subsection{Reality of Duality}

Two dimensions of reality run side by side --- the visible and the invisible. The earthly tabernacle was a copy of the heavenly (Exodus 25:40; Hebrews 8:5). The Levitical priests served ``the copy and shadow of the heavenly things.'' Both were real: the visible illustrated the invisible. Believers now live in both realms --- on earth yet seated with Christ in heavenly places (Ephesians 2:6).

Shadow $\rightarrow$ Pointer $\rightarrow$ Reality (Christ)

\textit{``But the substance is of Christ.''} (Colossians 2:17)

\begin{tabularx}{\textwidth}{llXl}
\toprule
\textbf{Natural Reality (Shadow)} & \textbf{Pointer / Function} & \textbf{Christ --- The Reality} & \textbf{Verse (NKJV)} \\
\midrule
Clothing & Covers / identity & Christ is our garment --- we put on Christ & Galatians 3:27 \\
Family & Belonging / relationship & Christ defines our true family & Matthew 12:50 \\
Land / Ground & Place to walk / live & Walk in Him --- He is our place & Colossians 2:6 \\
House / Dwelling & Where we live & Christ in us --- we in Him & Colossians 1:27; 1 Corinthians 1:30 \\
Roots / Soil & Stability / nourishment & Rooted in Him --- He is our life-source & Colossians 2:7 \\
Bread / Food & Nourishment / sustenance & Christ is the true Bread & John 6:32--35 \\
Light & Direction / clarity & Christ is the true Light & John 1:9; 8:12 \\
Vine & Source of fruitfulness & Christ is the true Vine & John 15:1 \\
Water & Cleansing / refreshing & Christ is Living Water & John 4:14 \\
Drink & Satisfaction / life & Christ is true Drink & John 6:55 \\
Seat / Chair (Seated) & Rest / authority & Seated in Christ --- He is our seat & Ephesians 2:6 \\
\bottomrule
\end{tabularx}

We blame the Jews for missing Christ by clinging to sacrifices; yet many Christians miss Him by clinging to His gifts. They held the shadows and lost the Substance; we hold the gifts and lose the Giver.

Christ is not an addition to life --- Christ is the Life (Colossians 3:4). Everything God gives is a pointer; Christ is the point.

\subsection{Divine Separation --- The Principle of Holiness}

From the beginning God divided --- light from darkness, land from sea, Israel from the nations, Levites from the tribes, the high priest from the priests. (Leviticus 20:26) Separation is the principle of holiness.

The Principle of Holiness (Summary)

\subsubsection{God's Divine Order}

Everything begins as common. From the common God divides the clean and unclean; from the clean He chooses what is holy. Common $\rightarrow$ Clean $\rightarrow$ Holy $\rightarrow$ God's presence.

\subsubsection{Meanings}

Common -- ordinary, not yet claimed.

Clean -- fit to approach God. Unclean -- unfit for His presence (declared so by God's sovereignty, not by moral fault). Holy -- belonging to God, claimed and filled by Him.

\subsubsection{Purity and Sanctification}

Purity is freedom from mixture --- the inner truth of holiness (Matthew 5:8). Sanctification is the Spirit's work that makes holiness a living reality (Hebrews 13:12; John 17:17).

In Leviticus, sanctification was ritual and external; in Christ, it is spiritual and internal.

\subsubsection{Fulfilment in Christ}

In the Law: the unclean defiled the clean. In Christ: the Holy touches the unclean and makes them clean (Mark 1:41--42). Holiness now flows outward --- not separation alone but sanctifying presence.

\subsubsection{Summary Truth}

Holiness is not merely separation from sin but participation in God's own nature.

\subsection{Fourfold Meaning of Scripture (PaRDeS)}

\begin{tabularx}{\textwidth}{lllllX}
\toprule
\textbf{Level} & \textbf{Meaning} & \textbf{Focus} & \textbf{Example} & \textbf{In Christ} & \textbf{Description} \\
\midrule
\textbf{P'shat} & Literal sense & Historical & Animal brought (Leviticus 1:3) & He obeyed the Law perfectly. & What the Word says \\
\textbf{Remez} & Hint / type & Prophetic & Lamb $\rightarrow$ Christ (John 1:29) & He fulfilled every symbol and type. & What the Word points to \\
\textbf{Drash} & Moral lesson & Application & ``Be holy'' (Leviticus 19:2) & He taught the moral and spiritual heart of the Law (Matthew 5--7). & What the Word teaches \\
\textbf{Sod} & Hidden mystery & Spiritual & Veil $\rightarrow$ Christ's flesh (Hebrews 10:19--20) & He revealed the mystery hidden for ages --- ``Christ in you, the hope of glory.'' (Colossians 1:26--27) & What the Word reveals \\
\bottomrule
\end{tabularx}

``In Him are hidden all the treasures of wisdom and knowledge.'' (Colossians 2:3)

\subsection{Summary of Principles}

\begin{tabularx}{\textwidth}{lXl}
\toprule
\textbf{Principle} & \textbf{Meaning} & \textbf{Fulfilled in Christ} \\
\midrule
Shadow $\rightarrow$ Substance & Earthly pattern $\rightarrow$ Heavenly reality & Colossians 2:17 \\
Earthly $\rightarrow$ Heavenly & Visible mirrors invisible & Hebrews 9:11 \\
Division $\rightarrow$ Holiness & Separation reveals belonging & John 17:17 \\
Fourfold Meaning & Scripture in layers & Colossians 2:3 \\
\bottomrule
\end{tabularx}

\section{Understanding the Book of Leviticus Itself}

\subsection{Name and Meaning}

English: Leviticus = ``pertaining to the Levites.'' Hebrew: Vayikra (\heb{וַיִּקְרָא}) = ``And He called.'' (Leviticus 1:1)

\subsection{Context and Setting}

Location: At the foot of Mount Sinai.

Time: About one year after the Exodus.

Tabernacle completed; God speaks from within it.

\begin{tabularx}{\textwidth}{lXX}
\toprule
& \textbf{Exodus} & \textbf{Leviticus} \\
\midrule
1 & Exodus offers pardon & Leviticus offers purity \\
2 & God's approach to man & Man's approach to God \\
3 & Christ is Savior & Christ is Sanctifier \\
4 & Mans guilt is prominent & Mans defilement is prominent \\
5 & God speaks from the mount & God speaks from the tabernacle \\
6 & Man is made nigh to God & Man is kept nigh to God \\
7 & Ends with God's glory filling the Tabernacle & Begins with God calling man to draw near \\
8 & God dwelling among His people. & God teaching His people how to dwell with Him. \\
9 & Distance & Fellowship \\
10 & Fear & Access \\
\bottomrule
\end{tabularx}

\subsection{Position in the Pentateuch}

\textbf{Leviticus is the heart of the Torah --- and its heart word is holiness.}

\begin{scripture}
``You shall be holy, for I the LORD your God am holy.''
\scriptureref{Leviticus 19:2}
\end{scripture}

\begin{tabularx}{\textwidth}{lXX}
\toprule
\textbf{Book} & \textbf{Emphasis} & \textbf{Revelation of Christ} \\
\midrule
Genesis & Man ruined & Christ is promised \\
Exodus & Man redeemed & Christ the Deliverer \\
Leviticus & Man reconciled, worshiping & Christ the Mediator, the priest \\
Numbers & Man tested - rebellion & Christ the Guide \\
Deuteronomy & Man instructed, renewed & Christ the Prophet-Teacher \\
\bottomrule
\end{tabularx}

\subsection{Purpose and Theme}

\textbf{Question:} ``How can a sinful people dwell in the midst of a holy God?''

\textbf{Chs 1--16} $\rightarrow$ Instruction for Worship -- the way to God

\textbf{Chs 17--27} $\rightarrow$ Instruction for Holiness -- the walk with God

Access through blood $\rightarrow$ Fellowship through holiness.

\subsection{Structure of the Book}

\begin{tabularx}{\textwidth}{llXX}
\toprule
\textbf{Section} & \textbf{Chapters} & \textbf{Theme} & \textbf{Christological Focus} \\
\midrule
1--7 & Offerings & How to approach God & Christ our Sacrifice \\
8--10 & Priesthood & Who may approach & Christ our High Priest \\
11--15 & Cleanness & What defiles / purifies & Christ our Cleanser \\
16 & Day of Atonement & Way of access & Christ our Atonement \\
17--22 & Holiness & Walk and service & Christ our Sanctifier \\
23--25 & Feasts / Sabbaths & God's calendar & Christ our Fulfillment \\
26--27 & Covenant / Vows & Outcome of holiness & Christ our Lord and Reward \\
\bottomrule
\end{tabularx}

Begins with \textbf{sacrifice} and ends with \textbf{sanctification} --- the journey from the Cross to Consecration.

\subsection{Key Verse and Idea}

\textit{``Be ye holy, for I am holy.''} (Leviticus 11:44; 19:2; 20:7). \textbf{Theme:} Holiness through Atonement.

\subsection{Spiritual Message}

Redemption is not the end but the beginning (Exodus 25:8). Leviticus shows what kind of people can live in God's presence. \textit{``You shall be to Me a kingdom of priests and a holy nation.''} (Exodus 19:6). \textit{``You are a chosen generation, a royal priesthood''} (1 Peter 2:9). Leviticus reveals \textbf{Christ as our approach, Mediator, and Holiness.}

\subsection{Key to Understanding Leviticus}

Read with spiritual eyes: the altar, priest, blood, and feasts all point to Christ---the Lamb, the High Priest, the Sanctuary, the Presence of God.

\subsection{Summary Reflection}

Exodus ends with glory \textbf{filling} the Tabernacle; Leviticus begins with a voice \textbf{calling} from it. God calls not from the mountain of fear but from the tent of fellowship. The message is not ``Stay away,'' but \textbf{``Come near.''} The word ``come near'' defines access to God. Leviticus is precisely about how a sinful people may draw near to a holy God.

The offerings (qorban) themselves come from the same root qrb, meaning ``something brought near.'' The priesthood's role was to mediate nearness. The sacrifices symbolized approach through substitution and cleansing.

\textbf{(draw near, approach, bring -- 80-85 times)}

\section{God's Justice System --- The Law and the Sacrifice}

Leviticus reveals that God's justice system stands upon two unchanging pillars: the Law and the Sacrifice.

The Law declares the standard of God's holiness. The Sacrifice provides the solution for man's sinfulness. Without the Law, sin is unknown; without the Sacrifice, sin is unforgiven. Together they reveal the perfect harmony of righteousness and mercy.

\subsection{The Law --- The Holy Demand}

\textit{``The soul that sinneth, it shall die''} (Ezekiel 18:4). \textit{``Cursed is everyone who does not continue in all things written in the book of the law''} (Deuteronomy 27:26; Galatians 3:10). The Law exposes sin but cannot cleanse it; it judges but cannot justify. It reveals God's absolute holiness and our desperate need for a Redeemer. \textit{\textbf{``By the law is the knowledge of sin''}} \textbf{(Romans 3:20).}

\subsection{The Sacrifice --- The Merciful Provision}

\textit{``Without shedding of blood there is no remission''} \textbf{(Hebrews 9:22).} Where the Law condemns, the Sacrifice intercedes. The altar becomes the meeting place of justice and grace.

\textit{``Mercy and truth are met together; righteousness and peace have kissed each other''} (Psalm 85:10). Every offering in Leviticus declares that: The Law demands death for sin. The Sacrifice offers life through substitution.

\subsection{The Harmony of Law and Sacrifice in Christ}

At the Cross both aspects meet perfectly.

\begin{tabularx}{\textwidth}{lXX}
\toprule
\textbf{Aspect} & \textbf{Revealed in the Law} & \textbf{Fulfilled in Christ} \\
\midrule
Justice & The sinner must die (Ezekiel 18:4) & ``Christ died for our sins.'' (1 Corinthians 15:3) \\
Holiness & Requires obedience & ``He became obedient unto death.'' (Philippians 2:8) \\
Mercy & Accepts a substitute & ``The Lord laid on Him the iniquity of us all.'' (Isaiah 53:6) \\
Love & Provides a way back & ``God so loved the world that He gave His Son.'' (John 3:16) \\
\bottomrule
\end{tabularx}

The Cross is where the Law is satisfied and the Sacrifice is fulfilled. \textit{``Do we then make void the law through faith? God forbid: we establish the law''} (Romans 3:31).

\subsection{Summary}

The Law demands holiness; the Sacrifice provides it. The Law declares guilt; the Sacrifice removes it. The Law reveals sin; the Sacrifice redeems the sinner. The Law says, \textit{``The soul that sinneth shall die.''} The Sacrifice answers, \textit{``Christ died for the ungodly.''} (Romans 5:6)

Leviticus thus shows both the severity of divine justice and the sweetness of divine mercy --- and both meet perfectly in Jesus Christ\textbf{,} \textit{``the Lamb slain from the foundation of the world.''} \textbf{(Revelation 13:8)}

\section{The Continuity of Redemption --- From Eden to Calvary}

The sacrifice of Christ is not a new idea but the culmination of one eternal plan.

In Eden, God clothed Adam and Eve with skins --- innocent life for guilty life.

Abel offered a lamb and found acceptance (Genesis 4:4).

Noah built an altar after the flood (Genesis 8:20--21).

Abraham offered Isaac but received a ram in his stead (Genesis 22:13).

At Passover, Israel was redeemed by the blood of the lamb (Exodus 12:13).

At Sinai, blood sealed the covenant (Exodus 24:8). All these anticipated \textit{``the Lamb slain from the foundation of the world.''} (Revelation 13:8)

\begin{tabularx}{\textwidth}{XX}
\toprule
\textbf{Old Covenant Shadow} & \textbf{Fulfilment in Christ} \\
\midrule
Animal sacrifice & Perfect Lamb of God \\
High priest entering yearly & Christ entered once for all (Hebrews 9:12) \\
Blood on altar & Blood on the Cross \\
Earthly tabernacle & Heavenly sanctuary \\
\bottomrule
\end{tabularx}

\section{The Voice from the Tabernacle --- A Closing Reflection}

Leviticus opens not with thunder and fear but with a call of grace: \textit{``Now the Lord called to Moses''} (Leviticus 1:1). God who dwelt in unapproachable light now invites His people to draw near through blood and faith. It is the voice of relationship, not ritual --- a God who wants to dwell among His own.

The book that begins with sacrifice ends with sanctification; what begins at the altar ends in communion. So too, our salvation begins at the Cross and finds its goal in a life wholly given to God.

\begin{scripture}
``You shall be to Me a kingdom of priests and a holy nation''
\scriptureref{Exodus 19:6}
\end{scripture}

\begin{scripture}
``To Him who loved us and washed us from our sins in His own blood\ldots to Him be glory and dominion forever.''
\scriptureref{Revelation 1:5--6}
\end{scripture}

Leviticus is therefore not a book of rituals but a revelation of Christ --- the Holy One who calls, cleanses, and communes with His people.

\subsection{Why So Many Sacrifices?}

\begin{enumerate}
\item \textbf{Typological Progression} -- Each offering revealed a distinct truth about Christ.
\item \textbf{Pedagogical Purpose} -- A visual gospel teaching holiness and substitution.
\item \textbf{Protective Purpose} -- Prevented Israel from pagan innovation (Deuteronomy 12:30--31).
\item \textbf{Prophetic Purpose} -- Foreshadowed the once-for-all sacrifice of Christ (Hebrews 10:11--12).
\end{enumerate}
