\chapter{Christ in the Covenant, the Sabbath, and the Jubilee}

\textit{Theme: Rest, Redemption, and Restoration --- Christ the Lord of the Sabbath and the Proclaimer of Jubilee.}

\section{Introduction: The Rhythm of Divine Rest}

Leviticus 25--26 concludes the covenantal section by revealing three theological realities:

\begin{enumerate}
\item \textbf{Rest (Sabbath and Sabbatical Year)} --- rest from labor and trust in divine provision.
\item \textbf{Redemption (Jubilee)} --- liberty and restoration by grace.
\item \textbf{Relationship (Covenant Blessings)} --- fellowship with God conditioned upon obedience, fulfilled in Christ.
\end{enumerate}

The moral pulse of Leviticus beats in the rhythm of \textbf{``Rest and Release.''} Every seventh day, every seventh year, and every fiftieth year proclaimed that \textbf{God owns the land, the people, and time itself.}

\begin{scripture}
``The land is Mine; for you are strangers and sojourners with Me.''
\scriptureref{Leviticus 25:23}
\end{scripture}

In Christ, this rhythm reaches its eternal fulfillment:
\begin{itemize}
\item He is \textbf{the true Sabbath} --- rest for the soul (Matt.\ 11:28--29).
\item He is \textbf{the true Redeemer} --- who proclaims the acceptable year of the Lord (Luke 4:18--19).
\item He is \textbf{the true Covenant Mediator} --- securing everlasting blessings (Heb.\ 8:6).
\end{itemize}

\section{Hebrew Word Studies}

\begin{table}[h]
\centering
\small
\begin{tabularx}{\textwidth}{ll>{\raggedright\arraybackslash}p{2.2cm}>{\raggedright\arraybackslash}X>{\raggedright\arraybackslash}X}
\toprule
\textbf{Word} & \textbf{Transliteration} & \textbf{Meaning} & \textbf{Reference} & \textbf{Fulfillment} \\
\midrule
\heb{שַׁבָּת} & \textit{šabbāṯ} & Rest, cessation & Lev.\ 25:2 & Matt.\ 11:28 --- Christ our rest \\
\heb{שַׁבַּת שַׁבָּתוֹן} & \textit{šabbat šabbātôn} & Sabbath of solemn rest & Lev.\ 25:4 & Heb.\ 4:9 --- Eternal rest \\
\heb{יוֹבֵל} & \textit{yōvēl} & Jubilee, ram's horn, liberty & Lev.\ 25:10 & Luke 4:18 --- ``Proclaim liberty'' \\
\heb{גָּאַל} & \textit{gāʾal} & To redeem, buy back & Lev.\ 25:25 & Eph.\ 1:7 --- ``In Him we have redemption'' \\
\heb{בְּרִית} & \textit{berîṯ} & Covenant & Lev.\ 26:9 & Heb.\ 8:6 --- Better covenant \\
\heb{חֶסֶד} & \textit{ḥesed} & Covenant faithfulness & Lev.\ 26:42 & John 1:17 --- Grace and truth \\
\bottomrule
\end{tabularx}
\end{table}

These words together form the grammar of grace: \textbf{rest} (Sabbath), \textbf{release} (Jubilee), and \textbf{relationship} (covenant). All three converge in Christ, who redeems time, land, and people for God.

\section{The Sabbath Year --- Trusting God's Provision (Leviticus 25:1--7)}

Every seventh year, the land was to rest. The Israelites were forbidden to sow, reap, or prune; they lived on what God provided spontaneously.

\begin{scripture}
``The Sabbath of the land shall provide food for you.''
\scriptureref{Leviticus 25:6}
\end{scripture}

\begin{table}[h]
\centering
\begin{tabularx}{\textwidth}{lXX}
\toprule
\textbf{Principle} & \textbf{Meaning} & \textbf{Fulfillment in Christ} \\
\midrule
Land rest & Creation's rhythm restored & Christ, Lord of creation \\
Trust in divine supply & Dependence on God, not toil & ``Your Father knows what you need'' (Matt.\ 6:32) \\
Equality among people & All shared freely & Fellowship of the saints (Acts 2:44--45) \\
\bottomrule
\end{tabularx}
\end{table}

The Sabbath year was not laziness but faith expressed in rest --- the outward picture of inward confidence in God. In Christ, this rest becomes spiritual reality:

\begin{scripture}
``We who have believed enter into rest.''
\scriptureref{Hebrews 4:3}
\end{scripture}

\section{The Year of Jubilee --- The Trumpet of Liberty (Leviticus 25:8--55)}

\subsection{The Institution of Jubilee}

Every fiftieth year, after seven sabbatical cycles, the trumpet (\heb{יוֹבֵל} \textit{yōvēl}) sounded on the Day of Atonement, proclaiming liberty throughout the land.

\begin{scripture}
``You shall hallow the fiftieth year, and proclaim liberty throughout all the land to all its inhabitants.''
\scriptureref{Leviticus 25:10}
\end{scripture}

This became Israel's year of restoration --- debts forgiven, slaves freed, property returned.

\subsection{The Threefold Liberation}

\begin{table}[h]
\centering
\small
\begin{tabularx}{\textwidth}{l>{\raggedright\arraybackslash}X>{\raggedright\arraybackslash}X}
\toprule
\textbf{Area} & \textbf{Action} & \textbf{Spiritual Fulfillment} \\
\midrule
Debt & All debts cancelled & Forgiveness through Christ's blood (Eph.\ 1:7) \\
Slavery & All slaves released & Deliverance from sin's bondage (Rom.\ 6:18) \\
Inheritance & Land restored & Restoration of spiritual inheritance in Christ (Eph.\ 1:11) \\
\bottomrule
\end{tabularx}
\end{table}

The Jubilee trumpet blew on the Day of Atonement --- liberty rooted in atonement. True freedom flows from forgiveness.

\subsection{The Theology of Jubilee}

\begin{table}[h]
\centering
\small
\begin{tabularx}{\textwidth}{l>{\raggedright\arraybackslash}X>{\raggedright\arraybackslash}X}
\toprule
\textbf{Symbol} & \textbf{Meaning} & \textbf{Christological Truth} \\
\midrule
Trumpet blast & Announcement of grace & Gospel proclamation \\
Return to possession & Restoration of inheritance & ``Heirs of God'' (Rom.\ 8:17) \\
Release of servants & Redemption by grace & ``If the Son sets you free\ldots'' (John 8:36) \\
No sowing / reaping & Dependence on provision & Grace, not works \\
Liberty at Atonement & Freedom through blood & ``The acceptable year of the Lord'' (Luke 4:18--19) \\
\bottomrule
\end{tabularx}
\end{table}

The Jubilee was both social legislation and spiritual prophecy --- a foretaste of the Kingdom age, when creation itself will be delivered from bondage (Rom.\ 8:21).

\section{The Kinsman-Redeemer (Leviticus 25:23--55)}

The laws of redemption introduce one of the most beautiful titles of Christ --- the \textbf{Go'el} (Redeemer).

\begin{table}[h]
\centering
\small
\begin{tabularx}{\textwidth}{l>{\raggedright\arraybackslash}X>{\raggedright\arraybackslash}X}
\toprule
\textbf{Provision} & \textbf{Description} & \textbf{Fulfillment in Christ} \\
\midrule
Redemption of property (25:25) & A near relative could buy back lost land & Christ, our Brother, restores lost inheritance \\
Redemption of persons (25:47--49) & A kinsman could redeem enslaved family & Christ redeems us from slavery to sin \\
Conditioned on relationship & Only a near kinsman qualified & Christ became man to qualify as Redeemer (Heb.\ 2:14--15) \\
\bottomrule
\end{tabularx}
\end{table}

\begin{scripture}
``I know that my Redeemer (Go'el) lives.''
\scriptureref{Job 19:25}
\end{scripture}

\begin{scripture}
``In Him we have redemption through His blood.''
\scriptureref{Ephesians 1:7}
\end{scripture}

The book of Ruth later illustrates this principle --- Boaz as the Go'el foreshadows Christ, the Redeemer who marries His Bride and restores her inheritance.

\section{Chart --- Sabbath Year vs. Jubilee Year}

\begin{table}[h]
\centering
\begin{tabularx}{\textwidth}{lXX}
\toprule
\textbf{Feature} & \textbf{Sabbath Year} & \textbf{Jubilee Year} \\
\midrule
Frequency & Every 7th year & Every 50th year \\
Land & Rests & Returns to original owner \\
Debts & Suspended & Cancelled \\
Servants & Rest & Released \\
Spiritual meaning & Rest in God's provision & Restoration in God's grace \\
Fulfillment in Christ & Spiritual rest (Matt.\ 11:28) & Full redemption (Luke 4:18--19) \\
\bottomrule
\end{tabularx}
\end{table}

Together, these years show rest and restoration --- the essence of redemption. The Sabbath year teaches faith; the Jubilee year proclaims freedom.

\section{The Covenant Blessings and Curses (Leviticus 26)}

After the laws of rest and release, God outlines the conditions of covenant relationship --- obedience bringing blessing, rebellion bringing discipline.

\subsection{Blessings for Obedience (26:1--13)}

\begin{table}[h]
\centering
\small
\begin{tabularx}{\textwidth}{l>{\raggedright\arraybackslash}X>{\raggedright\arraybackslash}X}
\toprule
\textbf{Blessing} & \textbf{Description} & \textbf{Fulfillment in Christ} \\
\midrule
Rain in season & Fruitfulness and provision & Spiritual fruit (John 15:5) \\
Peace in the land & Security & Peace with God (Rom.\ 5:1) \\
Victory over enemies & Triumph & ``More than conquerors'' (Rom.\ 8:37) \\
God's presence & Tabernacle among them & ``God with us'' (Matt.\ 1:23) \\
Covenant confirmation & Fellowship & New covenant in His blood (Luke 22:20) \\
\bottomrule
\end{tabularx}
\end{table}

\begin{scripture}
``I will walk among you, and be your God, and you shall be My people.''
\scriptureref{Leviticus 26:12}
\end{scripture}

This verse anticipates Immanuel --- God with us, the incarnate fulfillment of covenant presence.

\subsection{Curses for Disobedience (26:14--39)}

When Israel disobeyed, the land itself would rest by force during exile (26:34--35). Every unkept Sabbath year would be repaid --- fulfilled in the Babylonian captivity (2 Chron.\ 36:21).

\begin{table}[h]
\centering
\small
\begin{tabularx}{\textwidth}{l>{\raggedright\arraybackslash}X>{\raggedright\arraybackslash}X}
\toprule
\textbf{Principle} & \textbf{Leviticus} & \textbf{Fulfillment} \\
\midrule
Discipline for sin & Lev.\ 26:18 & Heb.\ 12:6 --- ``Whom the Lord loves He chastens.'' \\
Exile and desolation & Lev.\ 26:33 & National judgment for unbelief \\
Confession and restoration & Lev.\ 26:40--45 & Israel's future repentance (Rom.\ 11:26) \\
\bottomrule
\end{tabularx}
\end{table}

Even in judgment, mercy prevailed:

\begin{scripture}
``Yet for all that\ldots I will not cast them away\ldots I will remember My covenant.''
\scriptureref{Leviticus 26:44--45}
\end{scripture}

This steadfast mercy (\heb{חֶסֶד} \textit{ḥesed}) anticipates the new covenant grace revealed in Christ.

\section{Chart --- The Old and New Covenants}

\begin{table}[h]
\centering
\begin{tabularx}{\textwidth}{lXX}
\toprule
\textbf{Aspect} & \textbf{Old Covenant (Leviticus 26)} & \textbf{New Covenant (Hebrews 8--10)} \\
\midrule
Basis & Law & Grace \\
Mediator & Moses & Christ \\
Sacrifice & Animal blood & Christ's blood \\
Scope & National (Israel) & Universal (Church) \\
Duration & Temporary & Eternal \\
Sign & Sabbath & Spirit (Eph.\ 1:13) \\
Promise & Earthly blessings & Spiritual blessings \\
Result & Conditional fellowship & Secure relationship \\
\bottomrule
\end{tabularx}
\end{table}

\begin{scripture}
``He is the mediator of a better covenant, established on better promises.''
\scriptureref{Hebrews 8:6}
\end{scripture}

\section{The Fulfillment in Christ: Rest, Release, and Relationship}

\begin{table}[h]
\centering
\small
\begin{tabularx}{\textwidth}{l>{\raggedright\arraybackslash}X>{\raggedright\arraybackslash}X}
\toprule
\textbf{Theme} & \textbf{Leviticus Shadow} & \textbf{Fulfillment in Christ} \\
\midrule
Rest & Sabbath year & ``Come unto Me and rest.'' (Matt.\ 11:28) \\
Release & Jubilee & ``Proclaim liberty to the captives.'' (Luke 4:18) \\
Redemption & Go'el (kinsman-redeemer) & ``In Him we have redemption.'' (Eph.\ 1:7) \\
Relationship & Covenant fellowship & ``Abide in Me and I in you.'' (John 15:4) \\
Restoration & Return of land & ``All things reconciled'' (Col.\ 1:20) \\
\bottomrule
\end{tabularx}
\end{table}

Thus, the final chapters of Leviticus culminate in Christ --- the Rest-Giver, the Redeemer, and the Restorer.

\section{Doctrinal Synthesis}

\begin{enumerate}
\item \textbf{Sabbath typifies salvation rest.} Rest is not inactivity but trust --- faith ceases from self-effort.
\item \textbf{Jubilee prefigures full redemption.} Freedom from sin, debt, and bondage begins now and will be completed at Christ's return.
\item \textbf{The Covenant reveals God's faithfulness.} Even when His people fail, He remembers His covenant of mercy.
\item \textbf{Christ unites all three truths.} He gives rest to the weary, liberty to the captive, and covenant relationship to the redeemed.
\end{enumerate}

\begin{scripture}
``The Spirit of the Lord is upon Me\ldots to preach deliverance to the captives, to proclaim the acceptable year of the Lord.''
\scriptureref{Luke 4:18--19}
\end{scripture}

\section{Closing Meditation}

\begin{scripture}
``This shall be a Jubilee to you; and each of you shall return to his possession.''
\scriptureref{Leviticus 25:10}
\end{scripture}

\begin{scripture}
``If the Son therefore shall make you free, you shall be free indeed.''
\scriptureref{John 8:36}
\end{scripture}

Christ is our eternal Jubilee. He releases us from sin's slavery, restores our lost inheritance, and reconciles us to the Father. The Sabbath calls us to rest; the Jubilee calls us to rejoice; the Covenant calls us to relationship.

In Him, time itself finds its fulfillment, for He is both the Beginning and the End --- the Alpha and the Omega.
