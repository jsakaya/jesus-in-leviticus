\chapter{The Priesthood}

\begin{scripture}
``We have such a High Priest, who is seated at the right hand of the throne of the Majesty in the heavens.''
\scriptureref{Hebrews 8:1}
\end{scripture}

Aaron's priesthood was temporary and symbolic. Christ's priesthood is eternal and real --- the fulfillment of all Levitical shadows. Every act in the consecration of the priests (Leviticus 8) points to the Person and Work of Jesus Christ.

\section{The Ten Steps of the Priesthood -- Type and Antitype}

\begin{table}[htbp]
\centering
\small
\begin{tabularx}{\textwidth}{>{\raggedright\arraybackslash}p{3cm} l X X}
\toprule
\textbf{Step (Lev.\ 8)} & \textbf{Theme} & \textbf{Fulfilled in Christ} & \textbf{Meaning} \\
\midrule
1. Selection (v.2) & Chosen by God & ``A priest forever'' (Heb.\ 5:5--6) & Priesthood is by divine election \\
2. Approach (v.6) & Brought Near & Christ approached by His blood (Heb.\ 9:12) & Only those God brings near can serve \\
3. Washing (v.6) & Purity & Fulfilled all righteousness (Matt.\ 3:15) & True purity is inward holiness \\
4. Clothing (v.7) & Righteousness & Clothed in humanity and glory (Isa.\ 61:10) & His righteousness covers us \\
5. Anointing (v.12) & Spirit's Power & Anointed with the Spirit (Luke 4:18) & Spirit empowers priestly work \\
6. Substitution (v.14) & Sin Offering & Became the sin offering (1 Pet.\ 2:24) & He bore our guilt \\
7. Consecration (v.23) & Dedication & Perfect obedience in hearing, serving & Fully yielded to the Father \\
8. Service (v.27) & Ministry & ``My food is to do His will'' (John 4:34) & Hands filled with redemption \\
9. Separation (v.30) & Sanctification & ``For their sakes I sanctify Myself'' & Set apart for the Father \\
10. Participation (v.31) & Communion & ``He who eats My flesh abides in Me'' & Fellowship through His offering \\
\bottomrule
\end{tabularx}
\end{table}

\textbf{Summary:}

Aaron = the shadow. Christ = the substance.

Aaron was chosen, washed, clothed, anointed, consecrated, and fed in the tabernacle.

Christ was chosen before the foundation of the world, sinless, Spirit-anointed, consecrated through obedience unto death, and seated in the true tabernacle (Heb. 8:1--2).

\subsection{The Superiority of Christ's Priesthood -- Who He Is}

\begin{enumerate}
\item \textbf{Superior in Person} -- He is sinless and divine. (Heb. 7:26--28)

\item \textbf{Superior in Order} -- ``After the order of Melchizedek.'' (Heb. 7:17)

\item \textbf{Superior in Appointment} -- By divine oath, not law. (Heb. 7:21)

\item \textbf{Superior in Sympathy} -- He understands our weakness. (Heb. 4:15)
\end{enumerate}

Christ is the perfect, eternal, compassionate High Priest.

\subsection{The Supremacy of Christ's Priesthood -- Where and How He Ministers}

\begin{enumerate}
\item \textbf{Supreme in Sanctuary} -- Serves in heaven itself. (Heb. 9:24)

\item \textbf{Supreme in Sacrifice} -- Offered Himself once for all. (Heb. 9:12, 26; 10:10)

\item \textbf{Supreme in Covenant} -- Mediator of a new and better covenant. (Heb. 8:6--10)

\item \textbf{Supreme in Effectiveness} -- Perfected forever those who are sanctified. (Heb. 10:14)
\end{enumerate}

\subsection{The Sufficiency of Christ's Priesthood -- What He Accomplishes}

\begin{enumerate}
\item \textbf{For All Sin} -- His sacrifice satisfies completely. (Heb. 9:26)

\item \textbf{For All Time} -- ``He ever lives to make intercession.'' (Heb. 7:25)

\item \textbf{For All Access} -- ``Boldness to enter the Holiest.'' (Heb. 10:19--22)

\item \textbf{For All Needs} -- ``Able to save to the uttermost.'' (Heb. 7:25)
\end{enumerate}

\subsection{Summary Table}

\begin{table}[htbp]
\centering
\begin{tabularx}{\textwidth}{lXX}
\toprule
\textbf{Aspect} & \textbf{Levitical Priesthood} & \textbf{Christ's Priesthood} \\
\midrule
Order & Aaronic (temporary) & Melchizedek (eternal) \\
\midrule
Appointment & By law & By divine oath \\
\midrule
Character & Sinful men & Sinless Son of God \\
\midrule
Sacrifice & Many, repeated & One, perfect, final \\
\midrule
Sanctuary & Earthly & Heavenly \\
\midrule
Covenant & Old, fading & New, everlasting \\
\midrule
Effect & Could not perfect & Perfected forever \\
\midrule
Duration & Ends with death & Lives forever \\
\midrule
Access & Limited & Open to all believers \\
\midrule
Ministry & Shadow & Reality and fulfillment \\
\bottomrule
\end{tabularx}
\end{table}

\subsection{Devotional Reflection}

The priesthood of Aaron ended at the altar; the priesthood of Christ continues at the throne. Every drop of Levitical blood pointed to His Cross. Now every prayer of ours rises through His intercession. He is the Priest who never sleeps, the Advocate who never fails, and the Mediator who never ceases to love.

\section{The Three Appearances and the Three Eternals}

\textit{(Leviticus 9 and Hebrews 9)}

\begin{scripture}
``For Christ has not entered the holy places made with hands\ldots{} but into heaven itself, now to appear in the presence of God for us.''
\scriptureref{Hebrews 9:24}
\end{scripture}

After the consecration of the priesthood in Leviticus 8, chapter 9 describes the \textbf{public inauguration of their ministry}. In this chapter, the priest (Aaron) moves through \textbf{three distinct appearances}, each prefiguring the redemptive work of Christ --- past, present, and future.

\subsection{The Three Movements in Leviticus 9}

\begin{table}[htbp]
\centering
\small
\begin{tabularx}{\textwidth}{l l X X}
\toprule
\textbf{Lev.\ 9} & \textbf{Scene} & \textbf{Fulfilled in Christ} & \textbf{Significance} \\
\midrule
v.7 & At the Altar & ``He appeared to put away sin'' (Heb.\ 9:26) & \textbf{Past:} Christ's first appearing --- Calvary \\
v.23a & In the Tabernacle & ``Now to appear in God's presence for us'' (Heb.\ 9:24) & \textbf{Present:} Intercession in heaven \\
v.23b & Before the People & ``He shall appear a second time'' (Heb.\ 9:28) & \textbf{Future:} Glorious return \\
\bottomrule
\end{tabularx}
\end{table}

\textbf{Three Appearances = One Complete Redemption}

\begin{itemize}
\item \textit{He has appeared} --- to put away sin (Past)
\item \textit{He now appears} --- to intercede for us (Present)
\item \textit{He shall appear} --- to bring full salvation (Future)
\end{itemize}

\subsection{The Three Eternals in Hebrews 9}

Each appearance of Christ corresponds with an \textbf{eternal aspect} of His priestly work --- revealing its perfection and permanence.

\begin{table}[htbp]
\centering
\begin{tabularx}{\textwidth}{lXX}
\toprule
\textbf{Verse} & \textbf{Eternal Truth} & \textbf{Connection} \\
\midrule
\textbf{Hebrews 9:14} -- ``Through the eternal Spirit'' & \textbf{Eternal Spirit} & Power behind His sacrifice --- connects with \textbf{Heb. 9:26 / Lev. 9:7}, His past appearing. \\
\midrule
\textbf{Hebrews 9:12} -- ``Having obtained eternal redemption'' & \textbf{Eternal Redemption} & Result of His heavenly ministry --- connects with \textbf{Heb. 9:24 / Lev. 9:23a}, His present appearing. \\
\midrule
\textbf{Hebrews 9:15} -- ``The promise of eternal inheritance'' & \textbf{Eternal Inheritance} & Outcome of His return --- connects with \textbf{Heb. 9:28 / Lev. 9:23b}, His future appearing. \\
\bottomrule
\end{tabularx}
\end{table}

\subsection{Summary Chart}

\begin{table}[htbp]
\centering
\small
\begin{tabularx}{\textwidth}{l X l X}
\toprule
\textbf{Time} & \textbf{Christ's Appearance} & \textbf{Eternal Aspect} & \textbf{Type / Fulfilment} \\
\midrule
\textbf{Past} & He \textit{appeared} to put away sin & Eternal Spirit & Lev.\ 9:7 / Heb.\ 9:26 \\
\textbf{Present} & He \textit{appears} now for us & Eternal Redemption & Lev.\ 9:23a / Heb.\ 9:24 \\
\textbf{Future} & He \textit{shall appear} apart from sin & Eternal Inheritance & Lev.\ 9:23b / Heb.\ 9:28 \\
\bottomrule
\end{tabularx}
\end{table}

\subsection{Theological Insight}

\textbf{One Priest --- Three Appearances --- Three Eternals.}

In these three appearances, we see the entire span of salvation history:

\begin{itemize}
\item \textit{The Cross} --- finished work of atonement.

\item \textit{The Throne} --- ongoing work of intercession.

\item \textit{The Cloud} --- future work of glorification.
\end{itemize}

He appeared for our \textbf{atonement},

He appears for our \textbf{advocacy},

He shall appear for our \textbf{acclamation}.

\subsection{Devotional Reflection}

At the altar, He dealt with sin. In heaven, He represents us. Before the world, He will soon return in glory. The Levitical priest could only move from the altar to the tabernacle to the people. But Christ moves from \textbf{the Cross to the Throne to the Cloud} --- from \textbf{Calvary to the Heavenly Sanctuary to His Coming in Glory.}

\begin{scripture}
``Even so, come, Lord Jesus.''
\scriptureref{Revelation 22:20}
\end{scripture}
