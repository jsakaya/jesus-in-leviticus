\chapter{Christ in the Feasts of the Lord}

\textit{Theme: The Prophetic Calendar of Redemption --- Christ in Every Appointed Time.}

\section{Introduction: God's Calendar of Redemption}

Leviticus 23 presents the \textbf{Feasts of the Lord} --- not Israel's feasts, but \textbf{Yahweh's appointed times} (\heb{מוֹעֵדִים} \emph{môʿēdîm}). They are the divine blueprint of God's redemptive history --- every feast a prophetic picture of \textbf{Christ's person and work}.

\begin{scripture}
These are the appointed feasts of the LORD, holy convocations, which you shall proclaim in their seasons.
\end{scripture}
\scriptureref{Leviticus 23:4}

The Hebrew term \heb{מוֹעֵד} \emph{môʿēd} --- means an \textbf{appointed meeting} --- the time when heaven and earth meet in fellowship through redemption. Each feast, like a link in a chain, unfolds the \textbf{timeline of the Gospel}:

\begin{itemize}
\item Passover --- Christ's death
\item Unleavened Bread --- His burial
\item Firstfruits --- His resurrection
\item Pentecost --- The Spirit's outpouring
\item Trumpets --- His return for the Church
\item Day of Atonement --- Israel's national repentance
\item Tabernacles --- The Kingdom's fullness
\end{itemize}

\section{The Structure of Leviticus 23}

\begin{table}[htbp]
\centering
\caption{The Structure of Leviticus 23}
\begin{tabularx}{\textwidth}{l l X X}
\toprule
\textbf{Section} & \textbf{Feast} & \textbf{Month (Hebrew Calendar)} & \textbf{Symbolic Meaning} \\
\midrule
I & Sabbath & Weekly & Eternal rest in Christ \\
II & Passover & Nisan 14 & Redemption by blood \\
III & Unleavened Bread & Nisan 15--21 & Separation from sin \\
IV & Firstfruits & Nisan 17 & Resurrection life \\
V & Pentecost (Weeks) & Sivan 6 & Empowerment by the Spirit \\
VI & Trumpets & Tishri 1 & Regathering and resurrection \\
VII & Day of Atonement & Tishri 10 & National repentance \\
VIII & Tabernacles & Tishri 15--22 & God dwelling with men \\
\bottomrule
\end{tabularx}
\end{table}

The first four feasts were fulfilled \textbf{in Christ's first coming}; the last three will be fulfilled \textbf{in His second coming}. The gap between Pentecost and Trumpets pictures the \textbf{Church Age} --- the harvest time.

\section{The Hebrew Vocabulary of Appointed Times}

\begin{table}[htbp]
\centering
\caption{Hebrew Vocabulary of Appointed Times}
\begin{tabularx}{\textwidth}{l l X l X}
\toprule
\textbf{Word} & \textbf{Transliteration} & \textbf{Meaning} & \textbf{Reference} & \textbf{Fulfillment} \\
\midrule
\heb{מוֹעֵד} & \emph{môʿēd} & Appointed time & Lev 23:2 & Gal 4:4 -- ``Fullness of time'' \\
\heb{חַג} & \emph{ḥag} & Feast, festival & Lev 23:6 & John 7:37 -- ``Last day of the feast'' \\
\heb{שַׁבָּת} & \emph{šabbāṯ} & Rest, cessation & Lev 23:3 & Matt 11:28 -- ``I will give you rest'' \\
\heb{פֶּסַח} & \emph{pesaḥ} & Passover & Lev 23:5 & 1 Cor 5:7 -- ``Christ our Passover'' \\
\heb{שָׁבוּעוֹת} & \emph{šābuʿôt} & Weeks, Pentecost & Lev 23:15 & Acts 2:1 -- ``Day of Pentecost fully come'' \\
\heb{תְּרוּעָה} & \emph{tĕrûʿāh} & Blowing, trumpet blast & Lev 23:24 & 1 Thess 4:16 -- ``The trumpet of God'' \\
\heb{סֻכּוֹת} & \emph{sukkōṯ} & Booths, tabernacles & Lev 23:34 & John 1:14 -- ``The Word tabernacled among us'' \\
\bottomrule
\end{tabularx}
\end{table}

\section{The Feast of the Sabbath (Leviticus 23:1--3)}

The weekly Sabbath was the foundation of all feasts --- a continual reminder that \textbf{rest is found only in the Creator and Redeemer}.

\begin{table}[htbp]
\centering
\caption{The Feast of the Sabbath}
\begin{tabularx}{\textwidth}{l X X}
\toprule
\textbf{Symbol} & \textbf{Meaning} & \textbf{Fulfillment} \\
\midrule
Cessation of work & God's finished creation & Christ's finished redemption (John 19:30) \\
Seventh day & Completion & Eternal rest (Heb 4:9--10) \\
\bottomrule
\end{tabularx}
\end{table}

\begin{scripture}
Come unto Me\ldots{} and I will give you rest.
\end{scripture}
\scriptureref{Matthew 11:28}

Christ is the Sabbath Person --- \textbf{rest for the weary soul.}

\section{The Feast of Passover (Leviticus 23:4--5)}

\begin{itemize}
\item \textbf{Hebrew:} \heb{פֶּסַח} \emph{Pesaḥ} --- ``to pass over.''
\item \textbf{Date:} 14th day of Nisan.
\item \textbf{Meaning:} Deliverance from judgment through the blood of a lamb.
\end{itemize}

\begin{scripture}
When I see the blood, I will pass over you.
\end{scripture}
\scriptureref{Exodus 12:13}

\begin{table}[htbp]
\centering
\caption{The Feast of Passover}
\begin{tabularx}{\textwidth}{l X X}
\toprule
\textbf{Element} & \textbf{Symbolism} & \textbf{Fulfillment in Christ} \\
\midrule
Lamb without blemish & Innocence & ``Behold the Lamb of God'' (John 1:29) \\
Blood on doorposts & Substitution & ``Without shedding of blood\ldots'' (Heb 9:22) \\
Roasted with fire & Suffering & ``My God, why have You forsaken Me?'' \\
No bone broken & Integrity & John 19:36 \\
\bottomrule
\end{tabularx}
\end{table}

\subsection{Doctrinal Truth}

Passover is \textbf{Calvary in type}. Christ died on Passover day --- \emph{``Christ our Passover is sacrificed for us''} (1 Cor 5:7). The believer's deliverance from wrath begins here.

\section{The Feast of Unleavened Bread (Leviticus 23:6--8)}

Immediately following Passover, this feast lasted \textbf{seven days}, symbolizing \textbf{a continual walk of holiness}.

\begin{table}[htbp]
\centering
\caption{The Feast of Unleavened Bread}
\begin{tabularx}{\textwidth}{l X X}
\toprule
\textbf{Symbol} & \textbf{Meaning} & \textbf{Fulfillment} \\
\midrule
No leaven & Absence of corruption & Christ's sinless body (Heb 7:26) \\
Seven days & Complete devotion & Sanctified life of believers \\
Eating bread of purity & Fellowship with truth & ``Let us keep the feast\ldots{} with sincerity and truth'' (1 Cor 5:8) \\
\bottomrule
\end{tabularx}
\end{table}

Passover speaks of \textbf{deliverance}; Unleavened Bread speaks of \textbf{separation}. Christ's body ``saw no corruption'' (Acts 2:27); believers are called to live in the purity His death provides.

\section{The Feast of Firstfruits (Leviticus 23:9--14)}

Held on \textbf{the day after the Sabbath} following Passover --- the very morning of Christ's resurrection.

\begin{table}[htbp]
\centering
\caption{The Feast of Firstfruits}
\begin{tabularx}{\textwidth}{l X X}
\toprule
\textbf{Symbol} & \textbf{Meaning} & \textbf{Fulfillment} \\
\midrule
Sheaf of firstfruits & Beginning of harvest & Christ - firstfruit of resurrection (1 Cor 15:20) \\
Waving before the Lord & Presentation and acceptance & God accepted Christ's resurrection for us \\
No leavened bread & Sinlessness of resurrection life & New creation holiness \\
\bottomrule
\end{tabularx}
\end{table}

\begin{scripture}
Now is Christ risen from the dead, and become the firstfruits of them that slept.
\end{scripture}
\scriptureref{1 Corinthians 15:20}

Firstfruits marks \textbf{new creation life} --- the dawn of resurrection morning.

\section{The Feast of Weeks / Pentecost (Leviticus 23:15--22)}

\begin{itemize}
\item \textbf{Hebrew:} \heb{שָׁבוּעוֹת} \emph{Shābuʿôt} --- ``weeks''; Greek \emph{Pentēkostē} --- ``fiftieth.''
\item \textbf{Date:} Fifty days after Firstfruits.
\item \textbf{Meaning:} Thanksgiving for harvest; the completion of redemption.
\end{itemize}

\begin{table}[htbp]
\centering
\caption{The Feast of Weeks / Pentecost}
\begin{tabularx}{\textwidth}{l X X}
\toprule
\textbf{Element} & \textbf{Symbolism} & \textbf{Fulfillment} \\
\midrule
Two leavened loaves & Jew and Gentile, still imperfect & One new body, the Church \\
Fire and wind imagery & Divine presence & Holy Spirit descending (Acts 2) \\
New grain & New covenant community & Church born at Pentecost \\
\bottomrule
\end{tabularx}
\end{table}

\subsection{Doctrinal Truth}

The same Spirit who raised Christ now indwells His people. Pentecost is the \textbf{feast of fullness}, the harvest of resurrection life.

\section{The Feast of Trumpets (Leviticus 23:23--25)}

\begin{itemize}
\item \textbf{Hebrew:} \heb{יוֹם תְּרוּעָה} \emph{Yōm Tĕrûʿāh} --- ``Day of Blowing.''
\item \textbf{Date:} 1st of Tishri (the civil new year).
\item \textbf{Meaning:} Call to remembrance, repentance, and gathering.
\end{itemize}

\begin{table}[htbp]
\centering
\caption{The Feast of Trumpets}
\begin{tabularx}{\textwidth}{l X X}
\toprule
\textbf{Symbol} & \textbf{Prophetic Meaning} & \textbf{Fulfillment} \\
\midrule
Trumpet blast & Call to assemble & Rapture of the Church (1 Thess 4:16--17) \\
Awakening & Revival and repentance & Israel's awakening \\
New beginning & Renewal of covenant & Restoration of God's people \\
\bottomrule
\end{tabularx}
\end{table}

This feast looks forward to \textbf{the return of Christ} and the \textbf{regathering of Israel}, as well as the \textbf{resurrection of the saints}.

\section{The Day of Atonement (Leviticus 23:26--32)}

Already studied in session 4, here it reappears as the \textbf{sixth feast}, corresponding prophetically to \textbf{Israel's future repentance}.

\begin{table}[htbp]
\centering
\caption{The Day of Atonement}
\begin{tabularx}{\textwidth}{l l X}
\toprule
\textbf{Aspect} & \textbf{Leviticus} & \textbf{Prophetic Fulfillment} \\
\midrule
National humiliation & Lev 23:27 & Israel's mourning for the Pierced One (Zech 12:10) \\
Fasting, affliction & Lev 23:29 & ``They shall look on Him whom they pierced'' \\
High priest entering Holy of Holies & Lev 16; 23:27 & Christ appearing to Israel as Redeemer \\
Cleansing and forgiveness & Lev 16:30 & ``All Israel shall be saved'' (Rom 11:26) \\
\bottomrule
\end{tabularx}
\end{table}

The Day of Atonement, fulfilled once at Calvary, will one day be \textbf{applied nationally} when Israel recognizes her Messiah.

\section{The Feast of Tabernacles (Leviticus 23:33--44)}

\begin{itemize}
\item \textbf{Hebrew:} \heb{סֻכּוֹת} \emph{Sukkōṯ} --- booths, tents.
\item \textbf{Duration:} 15th--22nd of Tishri.
\item \textbf{Meaning:} Joyful remembrance of God's presence in the wilderness and \textbf{anticipation of His dwelling among men.}
\end{itemize}

\begin{table}[htbp]
\centering
\caption{The Feast of Tabernacles}
\begin{tabularx}{\textwidth}{l X X}
\toprule
\textbf{Element} & \textbf{Meaning} & \textbf{Fulfillment} \\
\midrule
Dwelling in booths & God's provision and protection & Incarnation --- ``The Word tabernacled among us'' (John 1:14) \\
Fruit of harvest & Celebration of ingathering & Joy of the Kingdom age \\
Water ceremony (later Jewish custom) & Outpouring of Spirit & ``If anyone thirsts\ldots{} rivers of living water'' (John 7:37--39) \\
Light ceremony & Illumination of God's glory & ``I am the Light of the world'' (John 8:12) \\
\bottomrule
\end{tabularx}
\end{table}

\subsection{Prophetic Vision}

Tabernacles points to the \textbf{Millennial Reign of Christ}, when the true Immanuel (``God with us'') reigns in righteousness and peace.

\begin{scripture}
Behold, the tabernacle of God is with men.
\end{scripture}
\scriptureref{Revelation 21:3}

\section{Chart --- The Prophetic Calendar of Redemption}

\begin{table}[htbp]
\centering
\caption{The Prophetic Calendar of Redemption}
\begin{tabularx}{\textwidth}{l l X X X}
\toprule
\textbf{Feast} & \textbf{Hebrew Name} & \textbf{Historical Fulfillment} & \textbf{Christological Fulfillment} & \textbf{Prophetic Fulfillment} \\
\midrule
1. Passover & \emph{Pesaḥ} & Exodus deliverance & Death of Christ & Calvary (past) \\
2. Unleavened Bread & \emph{Maṣṣōt} & Wilderness separation & Sinless burial of Christ & Sanctified life \\
3. Firstfruits & \emph{Reʾšît} & First harvest & Resurrection & New creation \\
4. Pentecost & \emph{Shābuʿôt} & Law at Sinai & Spirit at Pentecost & Church Age \\
5. Trumpets & \emph{Tĕrûʿāh} & New year trumpet & Rapture / resurrection & Israel's awakening \\
6. Atonement & \emph{Kippurîm} & Annual cleansing & Cross & Israel's repentance \\
7. Tabernacles & \emph{Sukkōṯ} & Wilderness joy & God with us & Millennial kingdom \\
\bottomrule
\end{tabularx}
\end{table}

\section{The Theology of the Feasts}

\begin{enumerate}
\item \textbf{Redemption (Passover)} --- The Lamb slain.
\item \textbf{Separation (Unleavened Bread)} --- The old life removed.
\item \textbf{Resurrection (Firstfruits)} --- New life begun.
\item \textbf{Empowerment (Pentecost)} --- Spirit poured out.
\item \textbf{Regathering (Trumpets)} --- Saints summoned.
\item \textbf{Reconciliation (Atonement)} --- Israel restored.
\item \textbf{Rejoicing (Tabernacles)} --- God dwelling forever.
\end{enumerate}

These feasts are not random holy days but \textbf{a sevenfold panorama of the Gospel} --- from cross to crown.

\section{Christ the Center of All Appointed Times}

\begin{table}[htbp]
\centering
\caption{Christ the Center of All Appointed Times}
\begin{tabularx}{\textwidth}{l l X}
\toprule
\textbf{Divine Pattern} & \textbf{Leviticus} & \textbf{Fulfillment in Christ} \\
\midrule
``Holy convocations'' & 23:2 & Church, the gathered people of God \\
``In their seasons'' & 23:4 & ``In the fullness of time'' (Gal 4:4) \\
``Offerings made by fire'' & 23:8, 18 & The work of the Spirit applying the cross \\
``Sabbath rest'' & 23:3 & Eternal rest in Christ (Heb 4:9--10) \\
\bottomrule
\end{tabularx}
\end{table}

Every feast begins and ends with \textbf{rest} --- all divine activity starts and concludes with Christ's finished work.

\section{Doctrinal Synthesis}

\begin{enumerate}
\item \textbf{The Feasts reveal God's redemptive timeline.}\\
From Passover to Tabernacles, the Gospel is prefigured.

\item \textbf{Christ fulfills every feast.}\\
He is the Lamb, the Bread, the Firstfruits, the Lord of the Harvest, the Coming King.

\item \textbf{The Spirit continues the work of Pentecost.}\\
The Church lives between the feasts of Pentecost and Trumpets --- the harvest interval.

\item \textbf{Israel's story is not finished.}\\
Trumpets, Atonement, and Tabernacles await national and universal fulfillment.

\item \textbf{Eternal rest crowns the plan.}\\
The feasts begin with redemption and end with God dwelling among His people.
\end{enumerate}

\section{Key Hebrew Vocabulary Recap}

\begin{table}[htbp]
\centering
\caption{Key Hebrew Vocabulary Recap}
\begin{tabularx}{\textwidth}{l l X l X}
\toprule
\textbf{Word} & \textbf{Transliteration} & \textbf{Meaning} & \textbf{Reference} & \textbf{NT Fulfillment} \\
\midrule
\heb{מוֹעֵד} & \emph{môʿēd} & Appointed time & Lev 23:2 & Gal 4:4 \\
\heb{פֶּסַח} & \emph{pesaḥ} & Passover & Lev 23:5 & 1 Cor 5:7 \\
\heb{מַצּוֹת} & \emph{maṣṣōt} & Unleavened bread & Lev 23:6 & 1 Cor 5:8 \\
\heb{בִּכּוּרִים} & \emph{bikkurîm} & Firstfruits & Lev 23:10 & 1 Cor 15:20 \\
\heb{שָׁבוּעוֹת} & \emph{šābuʿôt} & Weeks / Pentecost & Lev 23:15 & Acts 2:1 \\
\heb{תְּרוּעָה} & \emph{tĕrûʿāh} & Trumpets & Lev 23:24 & 1 Thess 4:16 \\
\heb{כִּפֻּרִים} & \emph{kippurîm} & Atonement & Lev 23:27 & Heb 9:12 \\
\heb{סֻכּוֹת} & \emph{sukkōṯ} & Tabernacles & Lev 23:34 & Rev 21:3 \\
\bottomrule
\end{tabularx}
\end{table}

\section*{Closing Meditation}

\begin{scripture}
These are the feasts of the LORD\ldots{} which you shall proclaim to be holy convocations.
\end{scripture}
\scriptureref{Leviticus 23:37}

From the first lamb slain in Egypt to the final harvest joy of Tabernacles, \textbf{Christ is the center of God's calendar}. Every trumpet blast, every loaf of unleavened bread, every booth and every sheaf --- all speak of Him.

\begin{scripture}
In the fullness of time God sent forth His Son.
\end{scripture}
\scriptureref{Galatians 4:4}

\begin{scripture}
The Word became flesh and tabernacled among us.
\end{scripture}
\scriptureref{John 1:14}

The feasts begin with blood and end with glory --- the same pattern by which the believer walks: \textbf{from redemption to rest, from cross to crown.}
