\chapter{Christ in the Laws of Cleansing and Holiness}

\emph{Theme: "Be holy, for I am holy" --- Cleansing and sanctification fulfilled in Christ.}

\section{Introduction: The God Who Separates and Sanctifies}

After the atonement of Leviticus 16, the next great section (Leviticus 11--20) unfolds the principle of \textbf{holiness} --- what it means to live as a cleansed and separated people.
Atonement makes access possible; holiness makes fellowship continual.

\begin{scripture}
For I am the LORD your God: you shall therefore sanctify yourselves, and you shall be holy; for I am holy.
\end{scripture}
\scriptureref{Leviticus 11:44}

This call to holiness is repeated throughout Leviticus (11:44--45; 19:2; 20:7--8).
Holiness is not ceremonial isolation but \textbf{conformity to God's character}.
All the ceremonial distinctions --- foods, diseases, discharges, moral boundaries --- were shadows pointing to the \textbf{spiritual purity found in Christ}.

\begin{table}[htbp]
\centering
\caption{Holiness in Leviticus and Christ}
\begin{tabularx}{\textwidth}{l X X}
\toprule
\textbf{Aspect} & \textbf{Leviticus Theme} & \textbf{Fulfillment in Christ} \\
\midrule
Ceremonial cleansing & External purity & Internal cleansing by His Spirit \\
Physical separation & Distinction from nations & Separation unto God \\
Moral holiness & Reflecting God's character & Christ's life reproduced in believers \\
\bottomrule
\end{tabularx}
\end{table}

\section{The Vocabulary of Purity and Sin}

\begin{table}[htbp]
\centering
\caption{Hebrew Vocabulary of Purity}
\begin{tabularx}{\textwidth}{l l l X X}
\toprule
\textbf{Hebrew Word} & \textbf{Transliteration} & \textbf{Meaning} & \textbf{Reference} & \textbf{Fulfillment} \\
\midrule
\heb{טָהוֹר} \emph{ṭāhôr} & --- & Clean, pure & Lev 11:47 & Matt 5:8 -- "Pure in heart" \\
\heb{טָמֵא} \emph{ṭāmēʾ} & --- & Unclean, defiled & Lev 11:47 & Mark 1:40--45 -- Cleansing of the leper \\
\heb{חֵטְא / חַטָּאת} \emph{ḥēṭʾ / ḥaṭṭāʾt} & --- & Sin / sin-offering & Lev 4:3 & 2 Cor 5:21 \\
\heb{קָדַשׁ} \emph{qāḏaš} & --- & To sanctify & Lev 20:7 & John 17:19 -- "I sanctify Myself for them" \\
\heb{נָזַה} \emph{nāzāh} & --- & To sprinkle & Lev 14:7 & Heb 12:24 -- "Blood of sprinkling" \\
\bottomrule
\end{tabularx}
\end{table}

These words move from the physical realm to the moral and finally to the spiritual --- tracing the same journey from \textbf{law to grace} and from \textbf{shadow to substance}.

\section{Laws of Cleanness and Uncleanness (Leviticus 11--15)}

The sections from Leviticus 11--15 teach that God's people must discern between the clean and the unclean in every aspect of life. This does not mean that some animals or conditions were morally sinful; rather, they symbolized spiritual truths.

\subsection{Clean and Unclean Animals (Leviticus 11)}

\begin{table}[htbp]
\centering
\caption{Categories of Clean and Unclean Animals}
\begin{tabularx}{\textwidth}{l X X X}
\toprule
\textbf{Category} & \textbf{Criterion} & \textbf{Symbolism} & \textbf{Fulfillment} \\
\midrule
Land animals & Split hoof + chew the cud & Discernment + meditation & Walk and Word (Ps 1; Heb 5:14) \\
Sea creatures & Fins and scales & Movement and protection & Separation and endurance \\
Birds & Avoid scavengers & Reject defilement & Purity of mind \\
Insects & Winged but grounded & Heavenly calling with humility & The life led by the Spirit \\
\bottomrule
\end{tabularx}
\end{table}

\textbf{Typology:}
The distinction between clean and unclean points to \textbf{spiritual discernment} --- walking according to the Spirit, not the flesh.

\begin{scripture}
Touch not the unclean thing, and I will receive you.
\end{scripture}
\scriptureref{2 Corinthians 6:17}

\textbf{Christological Fulfillment:}

In Mark 7:18--19, Jesus declared all foods clean, showing that external distinctions had served their symbolic purpose. True cleanness comes from the \textbf{heart purified by faith} (Acts 15:9).

\subsection{Purification after Childbirth (Leviticus 12)}

Childbirth, though sacred, rendered a woman ceremonially unclean, symbolizing the \textbf{transmission of sin through human generation} (Ps 51:5). Purification required \textbf{a sin offering and a burnt offering}.

Fulfillment: The Virgin Birth breaks the chain of defilement --- \emph{"That holy thing which shall be born of you shall be called the Son of God."} (Luke 1:35)

Mary herself brought the required offering (Luke 2:22--24), acknowledging that even motherhood stands under redemption.

\subsection{Laws of Leprosy (Leviticus 13--14)}

Leprosy in Scripture is a vivid picture of sin:

\begin{itemize}
\item It begins unseen.
\item It spreads silently.
\item It defiles completely.
\item It separates from fellowship.
\end{itemize}

\begin{table}[htbp]
\centering
\caption{Leprosy as a Type of Sin}
\begin{tabularx}{\textwidth}{l X X}
\toprule
\textbf{Element} & \textbf{Levitical Type} & \textbf{Spiritual Truth} \\
\midrule
Priest's examination & Diagnosis of leprosy & The Word discerns sin \\
Isolation & Separation from camp & Sin separates from God \\
Garment and house leprosy & Contamination of environment & Sin corrupts community \\
Cleansing with blood and water (14:4--7) & Two birds --- one slain, one released & Death and resurrection of Christ \\
\bottomrule
\end{tabularx}
\end{table}

\begin{scripture}
He shall sprinkle upon him that is to be cleansed seven times.
\end{scripture}
\scriptureref{Leviticus 14:7}

\textbf{Fulfillment:} Christ cleanses the leper by touch (Mark 1:40--45), signifying that holiness is \textbf{contagious in Him}, not defilement. In His humanity, He touched corruption without becoming unclean, because \textbf{He is purity itself}.

\subsection{Bodily Discharges and Impurity (Leviticus 15)}

These regulations symbolized that sin flows from within man, defiling even the most intimate parts of life. Jesus interpreted them spiritually:

\begin{scripture}
What comes out of a man, that defiles him.
\end{scripture}
\scriptureref{Mark 7:20--23}

Thus Leviticus 15 anticipates the need for \textbf{inward cleansing} --- fulfilled in the new birth and sanctification by the Spirit (Titus 3:5).

\section{The Sanctity of Blood (Leviticus 17)}

\begin{scripture}
For the life (nephesh) of the flesh is in the blood, and I have given it to you upon the altar to make atonement for your souls.
\end{scripture}
\scriptureref{Leviticus 17:11}

This verse is the theological center of Leviticus.
Life belongs to God, and blood --- the carrier of life --- must be treated as sacred.

\begin{table}[htbp]
\centering
\caption{The Sanctity of Blood}
\begin{tabularx}{\textwidth}{l l X}
\toprule
\textbf{Principle} & \textbf{Leviticus} & \textbf{Fulfillment} \\
\midrule
Life in the blood & Lev 17:11 & "This is My blood of the new covenant" (Matt 26:28) \\
Blood prohibited for food & Lev 17:12 & Respect for life's sacredness \\
Blood for atonement & Lev 17:11 & "Without shedding of blood is no remission" (Heb 9:22) \\
\bottomrule
\end{tabularx}
\end{table}

Christ's blood is not a symbol but the \textbf{very life poured out} for redemption. It satisfies justice, cleanses conscience, and consecrates believers (1 Pet 1:2).

\section{Moral and Ethical Holiness (Leviticus 18--20)}

Chapters 18--20 move from ceremonial purity to moral purity --- from the \textbf{shadow} to the \textbf{substance} of holiness.

\subsection{Prohibitions (Leviticus 18)}

Forbidden sexual relations represent moral boundaries rooted in divine order. Israel's holiness was to distinguish them from Egypt and Canaan --- symbolic of the world's corruption.

In Christ, holiness is not law-keeping but \textbf{Spirit-led morality}:

\begin{scripture}
For this is the will of God, your sanctification.
\end{scripture}
\scriptureref{1 Thessalonians 4:3}

\subsection{Positive Commands (Leviticus 19)}

This chapter is sometimes called the \textbf{Levitical Sermon on the Mount}. Its repeated phrase, \emph{"I am the LORD,"} anchors every ethical command in divine character.

\begin{table}[htbp]
\centering
\caption{Commands in Leviticus 19 and Their Fulfillment}
\begin{tabularx}{\textwidth}{l l X}
\toprule
\textbf{Command} & \textbf{Reference} & \textbf{Christ's Fulfillment} \\
\midrule
Love your neighbor & Lev 19:18 & Matt 22:39 -- "The second is like it." \\
Honest measures & Lev 19:35--36 & Integrity of Christ \\
Respect for elders & Lev 19:32 & Honor in the Kingdom \\
Care for the poor & Lev 19:9--10 & Christ's compassion in action \\
\bottomrule
\end{tabularx}
\end{table}

\textbf{Principle:} True holiness expresses itself in justice, mercy, and love.

\subsection{Penalties for Sin (Leviticus 20)}

The chapter concludes with the \textbf{penalty for defilement} --- death or separation --- underscoring sin's seriousness.

\begin{scripture}
You shall therefore be holy to Me, for I the LORD am holy, and have separated you from the peoples, that you should be Mine.
\end{scripture}
\scriptureref{Leviticus 20:26}

\textbf{Christological Fulfillment:}
Christ bore these penalties in Himself. The holiness demanded by law was fulfilled and imparted through the Spirit's indwelling (Rom 8:3--4).

\section{Christ the Fulfillment of Cleansing}

\begin{table}[htbp]
\centering
\caption{Symbols of Cleansing Fulfilled in Christ}
\begin{tabularx}{\textwidth}{l l X}
\toprule
\textbf{Levitical Symbol} & \textbf{Meaning} & \textbf{Fulfilled in Christ} \\
\midrule
Water & Cleansing & "Born of water and Spirit" (John 3:5) \\
Blood & Atonement & "The blood of Jesus cleanses from all sin" (1 John 1:7) \\
Oil & Consecration & "Anointed with the Holy Spirit" (Acts 10:38) \\
Fire & Purification & "He will baptize with the Holy Spirit and fire" (Matt 3:11) \\
Priest's declaration & Assurance & "Your faith has made you clean" (Mark 5:34) \\
\bottomrule
\end{tabularx}
\end{table}

Cleansing in Leviticus was ritual and repeated; in Christ, it is \textbf{spiritual and complete}.
He is both \textbf{the purifier and the purity} itself.

Hebrews 9:13--14 ---
\begin{scripture}
If the blood of bulls and goats... sanctifies for the purification of the flesh, how much more shall the blood of Christ... cleanse your conscience from dead works to serve the living God.
\end{scripture}

\section{Holiness: The Goal of Cleansing}

Cleansing leads to consecration; holiness is not a condition we achieve but a \textbf{relationship we maintain}. The refrain "I am the LORD" in Leviticus 19 appears fifteen times --- God Himself is the measure and motive of holiness.

\begin{table}[htbp]
\centering
\caption{Stages of Holiness}
\begin{tabularx}{\textwidth}{l X X}
\toprule
\textbf{Stage} & \textbf{Leviticus Pattern} & \textbf{Christological Reality} \\
\midrule
Cleansing & Removal of defilement & Justification \\
Consecration & Dedication to God & Sanctification \\
Communion & Fellowship with God & Union with Christ \\
\bottomrule
\end{tabularx}
\end{table}

Holiness is not isolation from sinners but \textbf{participation in the divine nature} (2 Pet 1:4).

\section{Chart --- From External Purity to Internal Holiness}

\begin{table}[htbp]
\centering
\caption{From External Purity to Internal Holiness}
\begin{tabularx}{\textwidth}{l X X}
\toprule
\textbf{Levitical Realm} & \textbf{Symbolic Concern} & \textbf{Fulfillment in Christ} \\
\midrule
Foods (ch. 11) & Discernment & New heart discerning right and wrong (Heb 5:14) \\
Childbirth (ch. 12) & Transmission of life & New birth in Christ \\
Leprosy (ch. 13--14) & Defilement of sin & Cleansing by the Word \\
Bodily discharge (ch. 15) & Inward pollution & Renewal of heart by Spirit \\
Blood (ch. 17) & Life and atonement & Eternal life through His blood \\
Morality (ch. 18--20) & Behavior and character & Christ-likeness in conduct \\
\bottomrule
\end{tabularx}
\end{table}

All purity laws converge in the person of Jesus --- \textbf{the Clean touching the unclean to make them clean}.

\section{The Theology of Separation}

The Hebrew root \emph{badal} (\heb{בָּדַל}) --- "to separate" --- runs throughout these chapters.
God divides to sanctify. He separates light from darkness, Israel from nations, and clean from unclean.

In Christ, this separation becomes inward --- the new creation separated from the old.

\begin{scripture}
Therefore, if anyone is in Christ, he is a new creation.
\end{scripture}
\scriptureref{2 Corinthians 5:17}

Separation is not withdrawal from the world but \textbf{distinction within it} --- being in the world yet not of it (John 17:15--19).

\section{Doctrinal Synthesis}

\begin{enumerate}
\item \textbf{Holiness proceeds from atonement.} Cleansing is grounded in sacrifice (Lev 17:11).

\item \textbf{Purity is both positional and practical.} We are made clean in Christ and called to walk clean by the Spirit.

\item \textbf{God's holiness defines morality.} Ethics flow from worship.

\item \textbf{Christ fulfills both ceremonial and moral law.} He is the true clean One, the purifier of His people.

\item \textbf{The indwelling Spirit applies holiness.} What law commanded, grace enables.
\end{enumerate}

\begin{scripture}
Christ loved the Church and gave Himself for her, that He might sanctify and cleanse her with the washing of water by the word.
\end{scripture}
\scriptureref{Ephesians 5:25--26}

\section{Reflection Questions}

\begin{enumerate}
\item How does Levitical cleanness reveal the seriousness of sin and the grace of God?

\item In what ways does Christ's ministry to lepers, women, and sinners illustrate the fulfillment of these chapters?

\item What is the difference between being "separated from" and "separated unto"?

\item How can a believer maintain purity without legalism?

\item Why is holiness both the cause and the consequence of fellowship with God?
\end{enumerate}

\section{Key Hebrew Vocabulary Recap}

\begin{table}[htbp]
\centering
\caption{Key Hebrew Vocabulary}
\begin{tabularx}{\textwidth}{l l l l X}
\toprule
\textbf{Word} & \textbf{Transliteration} & \textbf{Meaning} & \textbf{Reference} & \textbf{NT Fulfillment} \\
\midrule
\heb{טָהוֹר} \emph{ṭāhôr} & --- & Clean & Lev 11:47 & Matt 5:8 \\
\heb{טָמֵא} \emph{ṭāmēʾ} & --- & Unclean & Lev 11:47 & Mark 7:23 \\
\heb{קָדַשׁ} \emph{qāḏaš} & --- & To sanctify & Lev 20:7 & John 17:19 \\
\heb{בָּדַל} \emph{badal} & --- & To separate & Lev 20:24 & 2 Cor 6:17 \\
\heb{דָּם} \emph{dām} & --- & Blood & Lev 17:11 & Heb 9:14 \\
\bottomrule
\end{tabularx}
\end{table}

\section*{Suggested Reading}

\begin{itemize}
\item Andrew Bonar, \emph{A Commentary on Leviticus} (chs. 11--20)
\item C. H. Mackintosh, \emph{Notes on Leviticus}
\item F. B. Meyer, \emph{Christ in the Levitical Purity Laws}
\item R. K. Harrison, \emph{Leviticus: An Introduction and Commentary}
\item T. Austin-Sparks, \emph{The Law of Separation and Union in Christ}
\end{itemize}

\section*{Closing Meditation}

\begin{scripture}
You shall be holy to Me, for I the LORD am holy, and have separated you from the peoples, that you should be Mine.
\end{scripture}
\scriptureref{Leviticus 20:26}

Holiness is not self-made purity but God's ownership expressed in daily life. The God who cleanses also claims. The blood that redeems also sanctifies. In Christ, every barrier --- moral, ceremonial, or spiritual --- is overcome. He is both \textbf{the Cleansing Fountain} and \textbf{the Holy Place}.

\begin{scripture}
Having these promises, beloved, let us cleanse ourselves from all defilement of flesh and spirit, perfecting holiness in the fear of God.
\end{scripture}
\scriptureref{2 Corinthians 7:1}
